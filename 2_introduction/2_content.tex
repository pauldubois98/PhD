\section{Context}
Cancer remains one of the leading causes of mortality worldwide, with its incidence projected to rise in the coming decades \cite{Quante2016,Smittenaar2016}.
As our understanding of cancer biology evolves and diagnostic techniques improve, the demand for effective and personalized treatment strategies continues to grow \cite{Howard2017}.
Radiotherapy has emerged as a cornerstone in cancer management, either as a standalone treatment or in combination with other modalities such as surgery, chemotherapy, and immunotherapy \cite{Rivoirard2015,Reynders2015}.
Radiotherapy leverages ionizing radiation to damage cancer cells' DNA, impeding their ability to proliferate and ultimately leading to cell death \cite{Yard2015}.
The efficacy of radiotherapy lies in its ability to deliver precise doses of radiation to tumor volumes while minimizing exposure to surrounding healthy tissues \cite{Malicki2012}.
This delicate balance between tumor control and normal tissue toxicity underscores the importance of treatment planning in radiotherapy \cite{Das2009}.

The advent of advanced imaging technologies \cite{Li_Zhang_2024}, coupled with sophisticated delivery systems like Multi-Leaf Collimator Linear Accelerators (MLC-LINACs), Tomotherapy units, and CyberKnife systems, has revolutionized the field of radiation oncology \cite{Klein_2009,Xia2001}.
These technological advancements have paved the way for highly conformal treatment techniques.
Intensity-Modulated Radiation Therapy (IMRT) and Volumetric Modulated Arc Therapy (VMAT) techniques offer unprecedented levels of dose sculpting, allowing for escalated doses to tumors while better sparing organs at risk \cite{Ng2018,Elith2011,Davidson2011}.
However, the increased complexity of modern radiotherapy techniques has led to a corresponding increase in the complexity of treatment planning \cite{Fraass2012,Robinson2008}.
The process of creating an optimal treatment plan involves multiple steps, including Beam Orientation Optimization (BOO) \cite{Pugachev2000,Pugachev2001}, Fluence Map Optimization (FMO) \cite{Lim2008,Romeijn_2004,Lee2006}, Leaf Sequencing (LS) \cite{Chen2003,Chen2005,Xia2002}, and (sometimes) Direct Aperture Optimization (DAO) \cite{Shepard2002,Earl_2003,Ahunbay2007}.
Each step requires careful consideration of numerous variables and constraints, making the planning process time-consuming and labor-intensive \cite{Wang2019}.

In this context, this thesis aims to explore and advance the radiotherapy treatment planning automation field, focusing on developing novel algorithms and methodologies to enhance the efficiency, quality, and consistency of treatment plans.
By building upon the foundational knowledge of radiotherapy physics, biology, and clinical workflow, we seek to contribute to the ongoing evolution of radiation oncology and, ultimately, to improve outcomes for cancer patients.

\section{Problematic}
Traditional manual radiotherapy planning procedures are inherently subjective and time-consuming.
The reliance on individual planner expertise often leads to variability in treatment plan quality\cite{Chung2008,Bohsung2005,Das2008,Williams2007}.
This treatment plan diversity can induce inconsistencies in patient care.
While efforts have been made to enhance consistency \cite{Bahm2011}, significant variability among planners and institutions persists.
There is a pressing need for more standardized planning approaches.
The automation of treatment planning processes presents a promising solution to these challenges.

Moreover, the time-intensive nature of manual optimization and the growing demand for radiotherapy services have created a pressing need to develop automated approaches to streamline the radiotherapy planning process.
Automation enables the treatment of more patients and facilitates exploring a broader range of treatment options.

By applying computational algorithms and artificial intelligence, automated planning systems offer the potential to enhance radiotherapy treatment delivery significantly.
These systems can improve patient throughput and resource allocation by reducing planning time and increasing departmental efficiency.
Automated planning can improve plan quality and consistency, reducing variability and ensuring optimal treatment outcomes.
Furthermore, the ability to enable rapid re-planning facilitates adaptive radiotherapy, allowing for adjustments to treatment plans in response to changes in tumor volume or patient anatomy.
Finally, automated systems can explore a more expansive solution space, potentially discovering novel and innovative planning strategies that may improve treatment outcomes.

However, the development and implementation of automated planning systems pose challenges.
These challenges include the creation of robust optimization algorithms, integrating with existing Treatment Planning Systems, and validating against current clinical standards.

\section{State of the Art}
\paragraph{Knowledge-based radiotherapy planning}
Knowledge-based radiotherapy planning (KBRP) represents an objective methodology for incorporating patient-specific data and historical experience into the treatment planning process \cite{Nwankwo_2014}.
By automating the optimization of KBRP, it is anticipated that a viable alternative to the current human-centric treatment planning paradigm can be established.
A prevalent KBRP approach involves leveraging a database of historical benchmark plans to learn patient-specific dosimetric parameters and generate new treatment plans.
Automated KBRP tools effectively set optimization parameters based on the desired dose-volume histogram.
Previous studies have reported notable dosimetric improvements in treatment plans generated by KBRP approaches compared to benchmark data, particularly regarding sparing organs at risk \cite{Fogliata2014,Tol2015}.

\paragraph{Conventional techniques}

\paragraph{Pareto surface exploration}
Previous research has explored Pareto optimal tradeoffs in various domains.
Gebru et al., for instance, investigated methods for evaluating Pareto optimality \cite{Gebru2023}.
Similarly, Cilla et al. conducted a comprehensive analysis of Pareto fronts \cite{Cilla2018}.
These studies provide valuable insights into the principles and techniques associated with Pareto optimization.
Craft et al. proposed a method for generating multiple Pareto optimal dose distributions, allowing clinicians to make informed decisions based on their specific preferences and clinical contexts \cite{Craft2007}

\section{Unsolved problems}

\section{Contribution}
