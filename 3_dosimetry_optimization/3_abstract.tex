Biological tissues are sensible to radiations in a non-linear manner \cite{Liu2003}, and slight variations in dose can have significant biological effects.
Organs have differing sensibilities to radiation, which increases further the difficulty in formulating the goals to achieve when designing a radiation dose.
Some organs can tolerate high cumulative doses if the radiation is well distributed.
In contrast, others may withstand high doses at localized points ("hot spots") but cannot handle large doses overall.
To address these differences, clinicians impose dose-volume histogram constraints in addition to the prescribed dose.
Although the ideal objective is to minimize or eliminate radiation exposure to organs, achieving $0\,\text{Gy}$ is impossible.
The necessity of finding compromises drives the need for advanced optimization techniques to generate fluence maps that best satisfy the medical constraints.
Therefore, various techniques can be used to calculate fluence maps (i.e., performing the critical fluence map optimization step).
In this chapter, we explore some fluence map optimization techniques.
% These techniques will be used in the remaining manuscript.
