This dissertation explores the automation of radiotherapy dosimetry, focusing on optimization techniques to improve treatment planning.
Dosimetry, a crucial step in radiotherapy, directly impacts both the efficacy and safety of cancer treatments.
We investigated various optimization methods and proposed three automation frameworks:
a partially automated approach that allows clinicians to make adjustments,
a fully automated reinforcement learning system that eliminates manual intervention,
and a deep dose-based framework that uses deep learning while providing flexibility for manual adjustments through target dose-volume histograms.

This concluding chapter summarizes the main findings and limitations of the present manuscript.
We end in a prospective assessment of the long-term implications for the field of radiotherapy.
This chapter aims to provide a comprehensive overview of the potential impact of our findings on the advancement of radiotherapy practices in the future.
