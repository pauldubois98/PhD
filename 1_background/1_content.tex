% Content of the chapter 1 of my PhD thesis
\section{Medical context}

This PhD thesis is about radiation therapy (RT) for cancer treatment.

\subsection{About cancer}

Cancer is a complex disease that can affect many parts of the body.
It is characterized by the uncontrolled growth of cells that can invade and destroy surrounding tissues.
Cancer is a leading cause of death worldwide.
The World Health Organization (WHO) estimated 20 million new cancer cases in 2022, and 9.6 million deaths linked to cancer in 2022 \cite{who_cancer2022}.
Cancer touches about 20\% of the population, and is responsible for 1 in 10 deaths.

\paragraph{Cancer markers}
There are several cancer markers.
Cancer cells proliferate uncontrollably.
They also reprogram cellular metabolism to support their growth \cite{Chammas2013}.
They can also stop cell growth arrest mechanisms.
They usually manage to evade apoptosis (programmed cell death).
Cancer cells can escape the immune system, and change their cellular response phenotypic via plasticity.
At some point, cancer cells can get the ability to undergo a sufficient number of successive cell cycles of growth and division to generate macroscopic tumors.
To support their growth, they create new blood vessels to get nutrients.
Finally, they can escape and form metastasis, and will eventually provoke senescence.

\paragraph{Conditions leading to cancer}
Cancer is a complex disease.
First, cancer is caused by mutations in the DNA.
These mutations can be inherited or acquired.
Second, cancer is embraced by epigenetic reprogramming, i.e., changes in gene expression (that are not caused by changes in the DNA sequence).
Third, cancer is often associated with an inflammatory context, inflammation can promote cancer growth and spread.
Finally, cancer is often associated with a disruption of the microbiota (the microbial community living in and on the human body).
This disruption can promote cancer growth and spread.

\paragraph{Phases of cancer}
Cancer develops in several phases.
\subparagraph{Initiation}
The first phase is initiation: a normal cell is transformed into a cancer cell.
This transformation is caused by mutations in the DNA.
\subparagraph{Promotion}
The second phase is promotion or "tumorigenesis".
During this phase, the cancer cell grows and divides uncontrollably to form a cluster of cells called a tumor.
This growth is promoted by changes in gene expression and other factors \cite{Witsch2010}.
It may also create new blood vessels to get nutrients and oxygen.
\subparagraph{Evolution}
The final phase is evolution.
The tumor will first grow locally, then regionally, invading and damaging surrounding tissues.
Finally, the cancer cell will spread to other body parts, forming metastasis.
Metastasis is the leading cause of death in cancer patients \cite{Steeg2006}.

\subparagraph{Cancer stages}
Cancer is classified into stages \cite{cancerresearchuk2023}.
\begin{itemize}
	\item Stage 0: 'in situ neoplasm'; it means a group of abnormal cells in an area of the body. The cells may develop into cancer at some time in the future.
	\item Stage 1: the cancer is small and contained within the organ it started in.
	\item Stage 2: the tumor is larger than in stage 1, but cancer hasn't started to spread into the surrounding tissues.
	\item Stage 3: the cancer is larger, it has started to spread into surrounding tissues and cancer cells in the lymph nodes nearby.
	\item Stage 4: the cancer has spread from where it started to another body organ. This is also called secondary or metastatic cancer.
\end{itemize}
Doctors use the TNM system to describe the stage of the cancer \cite{Brierley2016}.
\begin{itemize}
	\item [T] stands for the size of the Tumour;
	It can be 1, 2, 3, or 4, with 1 being small and 4 large.
	\item [N] stands for the number of lymph Nodes affected;
	It can be between 0 and 3.
	0 means that there are no lymph nodes containing cancer cells; 3 means many lymph nodes containing cancer cells.
	\item [M] stands for the existence of metastasis in another part of the body.
	It can be 0 (no spread) or 1 (the cancer has spread).
\end{itemize}

\paragraph{Most common cancers}
According to the WHO, the most common cancers are lung, breast, colorectal, prostate, skin, and stomach cancer.
This thesis mainly focus on prostate cancer, which is among the most common ones.

\paragraph{Risk factors}
Tobacco use, alcohol consumption, unhealthy diet, physical inactivity and air pollution are risk factors for other cancer types.
However, the main risk factor for prostate cancer is age.
Thus, it touches all social population evenly and is un-avoidable.

\subsection{Treatment types}

There are three main types of cancer treatment: surgery, radiation therapy, and chemotherapy.
The choice of treatment depends on the type and stage of cancer, the patient's age and general health, and other factors.

\paragraph{Surgery}
Surgery is the most effective cancer treatment.
It involves removing the tumor and surrounding tissue.
Surgery is often used to treat early-stage cancer that has not spread to other parts of the body.
For surgery to be possible, the tumor must be located in a place that the surgeon can easily access.
Surgery can be followed by other treatments, such as radiation therapy or chemotherapy, to kill any remaining cancer cells.
% give TNM / stage of cancer where surgery is possible
% give success rate of surgery

\subparagraph{Advantages}
Surgery is curative, meaning that cancer is completely removed, and the patient can "forget" about it.
It is also a local treatment, hence having limited side effects on the body.
Finally, only one session is needed.

\subparagraph{Disadvantages}
Surgery is invasive, and can be painful.
However, the main disadvantage, is that it can only be used for localized cancer (with no metastasis) and is accessible to the surgeon.

\paragraph{Chemotherapy}
Chemotherapy is a treatment that uses drugs to kill cancer cells.
It is systemic, meaning it can reach cancer cells anywhere in the body.
Therefore, it usually has strong side effects.
Chemotherapy is often used to treat cancer that has spread to multiple parts of the body (i.e., metastatic cancer).

Depending on how advanced the cancer is, chemotherapy can be used to cure, control, or relieve symptoms (palliation).
% give TNM / stage of cancer where each type of chemo is possible
% give success rate as well as side effects expected

\subparagraph{Advantages}
Chemotherapy can be used to treat cancer that has spread to multiple parts of the body.
It can also be used to relieve symptoms and improve quality of life.

\subparagraph{Disadvantages}
Chemotherapy is a heavy treatment, with strong side effects.
It can also weaken the immune system, making the patient more susceptible to infections.
Finally, it can be expensive.

\paragraph{Radiation therapy}
Radiation therapy is a treatment that uses high-energy radiation to kill cancer cells.
It is semi-local, meaning that it only affects the tumor, and the tissues traversed by the radiation beams.
Radioation therapy is curative most of the time.
It can be used alone or in combination with other treatments.
% give TNM / stage of cancer where each type of RT is possible
% give success rate of RT

Radiation therapy can be delivered in two ways: external radiation therapy and internal radiation therapy.
External radiation therapy uses a machine to deliver radiation to the tumor from outside the body.
Internal radiation therapy uses radioactive materials placed directly into or near the tumor.
This thesis focuses on external radiation therapy.

\subparagraph{Advantages}
Radiation therapy is a non-invasive treatment, with limited side effects.
It is relatively localized, and can be used to treat cancers that are not accessible via surgery.

\subparagraph{Disadvantages}
Radiation therapy still targets a little bit of healthy cells.
Depending on the patient's response, it may cause side effects.

\paragraph{Other treatments}
Cancer research is very active, and new treatments are constantly being developed.
These treatments are often used in combination with others.

\subparagraph{Immunotherapy}
Immunotherapy is a treatment that uses the body's immune system to fight cancer.
It can boost or change how the immune system works to find and attack cancer cells.
It is a systemic treatment.

\subparagraph{Targeted therapy}
Targeted therapy is a treatment that uses drugs to target specific molecules that are involved in cancer growth.
It is a systemic treatment.

\subparagraph{Hormone therapy}
Hormones are proteins or substances made by the body that help control how specific cell types work.
Hormone therapy is a treatment that uses drugs to block or lower the amount of hormones in the body that are involved in cancer growth.
It is a systemic treatment.

\subparagraph{Stem cell transplant}
A stem cell transplant is a treatment that uses stem cells to replace cells that have been damaged or destroyed by cancer treatment.
It is a systemic treatment.

\section{Physics of Radiotherapy}

Radiation therapy uses high-energy radiation to kill cancer cells.

\subsection{Ionizing radiation}
Ionizing radiation has enough energy to remove tightly bound electrons from atoms, creating ions.
X-rays and gamma rays are both electromagnetic radiations that are ionizing and high-energy photons.
Some particle radiations, such as particles, beta particles, and neutrons, are also ionizing, but radiotherapy uses photon radiations.

X-rays are produced by accelerating electrons to collide with a target material and are used in medical imaging, as well as in (external) radiation therapy.
In contrast, gamma rays originate from the radioactive decay of specific atomic nuclei and are used in (internal) radiation therapy. 

Because ionizing radiation therapy can damage the DNA in cells and lead to cell death, it is used in radiation therapy for treating cancer.

\subsubsection{Photon interactions}
Photon-matter interactions within an absorbing medium undergo a stochastic processes.
Several interactions are possible for photons.
Some random interactions generate secondary ionizing particles.
It is these secondary particles that deposit the energy in the medium.

\paragraph{Photoelectric effect}
The photoelectric effect is the process by which an atom absorbs a photon, and an electron is ejected from the atom.
The photon ceases to exist, and its energy is transferred to the electron.
The ejected electron is called a photoelectron.
The photoelectron can ionize other atoms, leading to the creation of secondary electrons.
The photoelectric effect is the dominant interaction for low-energy photons.

\paragraph{Compton scattering}
Compton scattering is the process by which an atom scatters a photon, and an electron is ejected from the atom.
The photon is scattered at an angle, and part of its energy is transferred to the electron.
The scattered photon is called a Compton electron.
The Compton electron can ionize other atoms, leading to the creation of secondary electrons.
Compton scattering is the dominant interaction for medium-energy photons.

\paragraph{Pair production}
Pair production is the process by which an atomic nucleus absorbs a photon and creates an electron-positron pair.
The photon ceases to exist, and its energy is transferred to the electron-positron pair.
The electron and positron can ionize other atoms, leading to the creation of secondary electrons.
Pair production is the dominant interaction for high-energy photons.

% add graphic of photon interactions
% Pair production, the photoelectric effect, and Compton scattering
% constitute the primary interaction mechanisms observed in the megavoltage energy range.

\paragraph{Photon attenuation}
The photon beam will be attenuated as it passes through the medium, and the intensity of the beam will decrease.
The dose deposition in the medium is proportional to the intensity of the photon beam.
The attenuation of the beam follows an exponential law concerning the depth of the medium traversed (Lambert-Beer law) \cite{Beer1852}:
$$I(x) = I_0 \exp(-\mu x)$$
where $I$ is the intensity of the photon beam after passing through a thickness $x$ of the medium,
$I_0$ is the initial intensity of the photon beam,
and $\mu$ is the attenuation coefficient of the medium.

% \paragraph{Non-ionizing radiation}
% Non-ionizing radiation is radiation that does not have enough energy to remove tightly bound electrons from atoms.
% This type of radiation is not used in radiation therapy.

\subsection{Biological effect on cells}
Ionizing radiation can damage the cells, leading to cell death in various ways.

\paragraph{Radiation effects on DNA}
Ionizing radiation damages the DNA in cells.
This leads to cell apoptosis, necrosis, or senescence.
Radiation induces DNA damage through both direct and indirect mechanisms:
Directly, it causes single-strand breaks (SSBs), double-strand breaks (DSBs), DNA crosslinks, and DNA-protein crosslinks.
Indirectly, radiation generates reactive oxygen species (ROS) and reactive nitrogen species (RNS), which further contribute to DNA damage.

\subparagraph{DNA repair}
Cells have mechanisms to repair DNA damage.
There are several types of DNA repair mechanisms, including base excision repair (BER), nucleotide excision repair (NER), mismatch repair (MMR), and double-strand break repair (DSBR).
Cancer cells often have defects in DNA repair mechanisms, making them more sensitive to radiation therapy \cite{Brierley2016}.
% DNA quality plot with DNA reparation only for healthy cells

\paragraph{Radiation affects the plasma membrane}
Radiation affects the cell membrane's biological properties by affecting its composition.
Radiations can also directly cause corrosive damage, damage to the membrane initiates signaling events that are important for the apoptotic response \cite{CohenJonathan1999}.
% Radiation regulates cell membrane signal transduction
% \cite{Brierley2016}

% \paragraph{Radiations and cell organelles performances}
% Radiation damages the endoplasmic reticulum
% IR-induced ribosomal changes
% Radiation affects the behavior of mitochondria 
% Irradiation-induced lysosomal damage
% \cite{Wang2018}

% \paragraph{Radiation alters the biological behavior of tumor cells & immune system}
% Effects of radiation on the proliferation scale
% Effects of radiation on the invasion and metastasis scale
% Effects of radiation on the cancer-promoting inflammation scale

% Ray-enhanced anti-CTLA-4 immunotherapy
% Achievements of radiation combined with anti–PD-1/PD-L1 immunotherapy
% TLR-mediated immunologic effects of RT


% %%%%%%%%%%%%%%%%%%%%%%%%%%%%%%%%%%%%%%%%%%%%%%%%%%%% %
\section{Patient Path}

% add a graphic
\subsection{Detection / diagnostic}
\subsection{RT Prescription}
\subsection{CT scan}
\subsection{Contouring}
\subsection{Treatment Planning}
\subsection{Irradiation Sessions}
\subsection{Follow-up}

\section{Machines}
% add graphic of usage per machine type, and per constructor
\subsection{Molds / 3D-RT}
\subsection{MLC-LINAC}
% modifiers: jaws & leaves
\subsection{Tomotherapy}
\subsection{CyberKnife}
\subsection{Brachytherapy}

\section{Irradiations techniques}
\subsection{IMRT}
\paragraph{Step and Shoot}
\paragraph{Sliding Window}
% => what we will focus on
\subsection{VMAT}

\section{Treatment Planning Systems}
\subsection{Manufacturer}
\paragraph{Eclipse (Varian)}
\paragraph{ONE | Planning (Elekta)}
\paragraph{Precision (Accuray)}
\subsection{Non-manufacturer}
\paragraph{RayStation (RaySearch)}
\paragraph{matRad (German Cancer Research Center - DKFZ)}
\paragraph{AutoPlan (TheraPanacea - coming soon)}

\section{Dosimetry steps}
\paragraph{Challenges}
% competeing goals
\subsection{BOO}
\subsection{FMO}
\subsection{LF}

\section{Simulation}
% https://oncologymedicalphysics.com/dose-calculation-algorithms/
