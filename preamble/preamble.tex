% title page
\def \englishTitle {Methods for automatization of radiotherapy dosimetry.}
\def \frenchTitle {Méthodes pour l'automatisation de la dosimetrie en radiothérapie.}

\def \specialty {Spécialité de doctorat: ...}
\def \doctoralSchool {École doctorale n$^{\circ}$ 573 Interfaces : matériaux, systèmes, usages, ED INTERFACE}
\def \graduateSchool {Graduate School: Sciences de l’Ingénierie et des Systèmes, SIS}

\def \researchUnits {
	\textbf{Radiothérapie} (Institut Régionale du Cancer de Montpellier), \textbf{Advanced Research} (TheraPanacea), et \textbf{Biomathématiques, MICS, Mathématiques et Informatique pour la Complexité et les Systèmes} (Université Paris-Saclay, CentraleSupélec)
}
\def \directors {
	\textbf{Nikos Paragios}, Professeur, et la co-direction de \textbf{Paul-Henry Cournède}, Professeur
}

\def \date {16 décembre 2024}
\def \author{Paul Raymond François DUBOIS}

\def \jury {
	\textbf{David \uppercase{Azria}} & Président\\
	Professeur des universités - Praticien hospitalier, Faculté de Médecine de Montpellier-Nîmes & \\
	\textbf{Daniela \uppercase{Thorwarth}} & Rapporteur \& Examinatrice \\
	Professeure des universités, Université de Tübingen & \\
	\textbf{Vincent \uppercase{Lepetit}} & Rapporteur \& Examinateur \\
	Professeur des universités, ENPC ParisTech & \\
	\textbf{Pascal \uppercase{Fenoglietto}} & Examinateur \\
	Docteur, Institut du Cancer de Montpellier & \\
	\textbf{Paul-Henry \uppercase{Cournède}} & Directeur \\
	Professeur de mathématiques, CentraleSupélec (Université Paris-Saclay) & \\
	\textbf{Nikos \uppercase{Paragios}} &  Co-Directeur \\
	Professeur émérite de mathématiques, CentraleSupélec (Université Paris-Saclay) & \\
}

% abstract page
\def \englishKeyWords {Mathematics, Artificial Intelligence, Radiotherapy}
\def \frenchKeyWords {Mathématiques, Intelligence Artificielle, Radiothérapie}

\def \englishAbstract {
	Radiotherapy dosimetry is critical in ensuring the precision and safety of cancer treatments.
	The complexity and variability of treatment planning necessitate advanced methodologies for automation and optimization.
	This thesis introduces novel approaches aimed at automating the radiotherapy dosimetry process.
	
	The research begins with developing a dosimetry engine, and comprehensively evaluating existing open-source optimization algorithms for treatment plannification.
	Then, this thesis analyzes the relationships between different treatment plans.
	This analysis leads to the proposal of a novel framework for multi-objective optimization and robust plan selection using graph theory.
	
	To further reduce the time required for radiotherapy planning, the thesis explores the application of reinforcement learning for dose optimization.
	The proposed system performs dosimetry for new patients by leveraging dose data from past patients.
	This fully automated method can adapt to clinical dependencies, reducing the need for manual fine-tuning and easing its adoption in practice.
	
	In addition, the thesis investigates the use of deep learning for dose prediction, proposing a series of models guided by target Dose Volume Histograms (DVH).
	This guidance facilitates the incorporation of guidelines into the deep-generated doses.
	Moreover, it allows a single model to be trained instead of one for each clinic.
	
	The contributions of this thesis represent advancements in radiotherapy dosimetry, paving the way for the development of a fully automated, clinically dependent treatment planning system designed to operate with minimal human intervention.
	These innovations could enhance clinical workflows, making radiotherapy more efficient.
}

\def \frenchAbstract {
	La dosimétrie en radiothérapie est essentielle pour garantir la précision et la sécurité des traitements contre le cancer.
	La complexité et la variabilité de la planification des traitements nécessitent des méthodologies avancées pour l'automatisation et l'optimisation.
	Cette thèse présente des approches novatrices visant à automatiser le processus de dosimétrie en radiothérapie.
	
	Cette thèse commence par le développement d'un moteur de dosimétrie et une évaluation approfondie des algorithmes d'optimisation open-source existants pour la planification des traitements.
	Ensuite, ce manuscrit analyse les relations entre différentes doses.
	Cette analyse conduit à la proposition d'un cadre novateur pour l'optimisation multi-objectif et la sélection robuste de plans à l'aide de la théorie des graphes.
	
	Afin de réduire davantage le temps nécessaire pour la planification en radiothérapie, la thèse explore l'application de l'apprentissage par renforcement pour l'optimisation des doses.
	Le système proposé réalise la dosimétrie pour de nouveaux patients en exploitant les données de dose des patients traités dans le passé.
	Cette méthode entièrement automatisée peut s'adapter aux pratiques de différentes cliniques, réduisant ainsi le besoin d'ajustements manuels et facilitant son adoption en pratique.
	
	De plus, la thèse examine l'utilisation de l'apprentissage profond pour la prédiction des doses, en proposant une série de modèles guidés par des Histogrammes Dose-Volume (DVH) cibles.
	Ce guidage orientation permet l'incorporation de directives lors de la génération de doses par les modèles.
	En outre, cette technique permet d'entraîner un seul modèle capable de s'adapter, plutôt qu'un modèle pour chaque clinique.
	
	Les contributions de cette thèse présentent des avancées dans la dosimétrie en radiothérapie, ouvrant la voie au développement d'un système de planification de traitement entièrement automatisé, s'adaptant aux contraintes cliniques.
	Ces innovations pourraient améliorer les flux de travail cliniques, en réduisant l'intervention humaine à un minimum, rendant la radiothérapie plus efficiente.
}