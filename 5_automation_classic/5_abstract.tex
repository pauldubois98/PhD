%Achieving optimal dose distribution in radiation therapy planning is a complex task, with contradicting goals. 
%Yet, this step is crucial with profound implications for patient treatment.
%The absence of universally agreed-upon constraints prioritization in radiation therapy planning complicates the definition of an optimal plan, requiring a delicate balance between multiple objectives.
%This balance usually ends up being done manually.

In radiation therapy, treatment planning involves balancing competing objectives.
The contradictory goals often lack universal prioritization.
Expert bias introduces variability in clinical practice, as the preferences of radiation oncologists and medical physicists shape treatment planning.
Traditionally, this balance is achieved through manual or semi-manual processes guided by the expertise of clinicians and planners.
This chapter explores approaches for fully automating the treatment planning process, focusing on classical optimization techniques constrained by dosimetric objectives. 

We will review established methods and propose new agents capable of optimizing dose without human interaction.
This innovative approach leverages previously defined dose distance metrics.
We aim to streamline and standardize the treatment planning workflow by fully automating the optimization process. 