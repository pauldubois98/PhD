\section{Medical context}

Cancer is a complex disease that can affect many part of the body.
It is characterized by the uncontrolled growth of cells that can invade and destroy surrounding tissues.
Cancer is a leading cause of death worldwide.
The World Health Organization (WHO) estimated 20 million new cancer cases in 2022, and 9.6 million deaths linked to cancer in 2022\cite{who_cancer2022}.
Cancer touches about 20\% of the population, and is responsible for 1 in 10 deaths.

\paragraph{Cancer markers}
There are several cancer markers.
Cancer cells proliferate uncontrollably.
They also reprogram cellular metabolism to support their growth.
They can also stop cell growth arrest mechanisms.
They usually manage to evade apoptosis (the programmed cell death).
Cancer cells can escape the immune system, and change their cellular response phenotypic via plasticity.
At some point, cancer cell can get the ability to undergo a sufficient number of successive cell cycles of growth and division to generate macroscopic tumors.
To support their growth, they create new blood vessels to get nutriments.
Finally, they can escape and form metastasis, and will eventually provoke senescence.

\paragraph{Conditions leading to cancer}
Cancer is a complex disease.
First, cancer is caused by mutations in the DNA.
These mutations can be inherited or acquired.
Second, cancer embrased by an epigenetic reprogramming.
That is, changes in gene expression that are not caused by changes in the DNA sequence.
Third, cancer is often associated with an inflammatory context.
Inflammation can promote cancer growth and spread.
Finally, cancer is often associated with a disruption of the microbiota.
The microbiota is the community of microorganisms that live in and on the human body.
Disruption of the microbiota can promote cancer growth and spread.

\subsection{4 cancer conditions}
\begin{itemize}
	\item mutation
	\item epigenetic reprogramming
	\item inflammatory context
	\item disruption of microbiota
\end{itemize}
% => on what does RT applies?

\subsection{phases of cancer}
\paragraph{initiation}
\paragraph{promotion}
\paragraph{tumorigenesis + neoangiogenesis}
\paragraph{evolution (local, regional, metastasis)}
% => at what stage does RT applies?

\subsection{cancer classification:}
\paragraph{tumor, node, mestastasis}
% => at what stage does RT applies?
\paragraph{stages classification:}
\begin{enumerate}
	\item stage 0 which corresponds to a so-called in situ tumor
	\item stage 1 which corresponds to a single, small tumor
	\item stage 2 which corresponds to a larger local volume
	\item stage 3 which corresponds to invasion of the lymph nodes or surrounding tissues
	\item stage 4 which corresponds to a wider extension in the body in the form of metastases
\end{enumerate}
% => at what stage does RT applies?

\subsection{treatment types}
\paragraph{surgery}
\paragraph{RT}
% DNA quality plot with DNA reparation only for healthy cells
\paragraph{chemotherapy}
\paragraph{combination}

% %%%%%%%%%%%%%%%%%%%%%%%%%%%%%%%%%%%%%%%%%%%%%%%%%%%% %
\section{Patient Path}

% add a graphic
\subsection{Detection / diagnostic}
\subsection{RT Prescription}
\subsection{CT scan}
\subsection{Contouring}
\subsection{Treatment Planning}
\subsection{Irradiation Sessions}
\subsection{Follow-up}

\section{Machines}
% add graphic of usage per machine type, and per constructor
\subsection{Molds / 3D-RT}
\subsection{MLC-LINAC}
% modifiers: jaws & leaves
\subsection{Tomotherapy}
\subsection{CyberKnife}
\subsection{Brachytherapy}

\section{Irradiations techniques}
\subsection{IMRT}
\paragraph{Step and Shoot}
\paragraph{Sliding Window}
% => what we will focus on
\subsection{VMAT}

\section{Treatment Planning Systems}
\subsection{Manufacturer}
\paragraph{Eclipse (Varian)}
\paragraph{ONE | Planning (Elekta)}
\paragraph{Precision (Accuray)}
\subsection{Non-manufacturer}
\paragraph{RayStation (RaySearch)}
\paragraph{matRad (German Cancer Research Center - DKFZ)}
\paragraph{AutoPlan (TheraPanacea - coming soon)}

\section{Dosimetry steps}
\paragraph{Challenges}
% competeing goals
\subsection{BOO}
\subsection{FMO}
\subsection{LF}

\section{Simulation}
% https://oncologymedicalphysics.com/dose-calculation-algorithms/
