\section{Context}
Cancer remains one of the leading causes of mortality worldwide, with its incidence projected to rise in the coming decades.
As our understanding of cancer biology evolves and diagnostic techniques improve, the demand for effective and personalized treatment strategies continues to grow.
Radiotherapy has emerged as a cornerstone in cancer management, either as a standalone treatment or in combination with other modalities such as surgery, chemotherapy, and immunotherapy.
Radiotherapy leverages ionizing radiation to damage cancer cells' DNA, impeding their ability to proliferate and ultimately leading to cell death.
The efficacy of radiotherapy lies in its ability to deliver precise doses of radiation to tumor volumes while minimizing exposure to surrounding healthy tissues.
This delicate balance between tumor control and normal tissue toxicity underscores the importance of treatment planning in radiotherapy.

\section{Problematic}
Manual optim is time consuming; need to automate

\section{State of the Art}
\paragraph{Knowledge-based radiotherapy planning}

\paragraph{Conventional techniques}

\paragraph{Pareto surface exploration}

\section{Unsolved problems}

\section{Contribution}
