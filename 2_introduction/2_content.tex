\section{Context}
Cancer remains one of the leading causes of mortality worldwide, with its incidence projected to rise in the coming decades \cite{Quante2016,Smittenaar2016}.
As our understanding of cancer biology evolves and diagnostic techniques improve, the demand for effective and personalized treatment strategies continues to grow \cite{Howard2017}.
Radiotherapy has emerged as a cornerstone in cancer management, either as a standalone treatment or in combination with other modalities such as surgery, chemotherapy, and immunotherapy \cite{Rivoirard2015,Reynders2015}.
Radiotherapy leverages ionizing radiation to damage cancer cells' DNA, impeding their ability to proliferate and ultimately leading to cell death \cite{Yard2015}.
The efficacy of radiotherapy lies in its ability to deliver precise doses of radiation to tumor volumes while minimizing exposure to surrounding healthy tissues \cite{Malicki2012}.
This delicate balance between tumor control and normal tissue toxicity underscores the importance of treatment planning in radiotherapy \cite{Das2009}.

The advent of advanced imaging technologies \cite{Li_Zhang_2024}, coupled with sophisticated delivery systems like Multi-Leaf Collimator Linear Accelerators (MLC-LINACs), Tomotherapy units, and CyberKnife systems, has revolutionized the field of radiation oncology \cite{Klein_2009,Xia2001}.
These technological advancements have paved the way for highly conformal treatment techniques.
Intensity-Modulated Radiation Therapy (IMRT) and Volumetric Modulated Arc Therapy (VMAT) techniques offer unprecedented levels of dose sculpting, allowing for escalated doses to tumors while better sparing organs at risk \cite{Ng2018,Elith2011,Davidson2011}.
However, the increased complexity of modern radiotherapy techniques has led to a corresponding increase in the complexity of treatment planning \cite{Fraass2012,Robinson2008}.
The process of creating an optimal treatment plan involves multiple steps, including Beam Orientation Optimization (BOO) \cite{Pugachev2000,Pugachev2001}, Fluence Map Optimization (FMO) \cite{Lim2008,Romeijn_2004,Lee2006}, Leaf Sequencing (LS) \cite{Chen2003,Chen2005,Xia2002}, and (sometimes) Direct Aperture Optimization (DAO) \cite{Shepard2002,Earl_2003,Ahunbay2007}.
Each step requires careful consideration of numerous variables and constraints, making the planning process time-consuming and labor-intensive \cite{Wang2019}.

In this context, this thesis aims to explore and advance the radiotherapy treatment planning automation field, focusing on developing novel algorithms and methodologies to enhance treatment plans' efficiency, quality, and consistency.
By building upon the foundational knowledge of radiotherapy physics, biology, and clinical workflow, we seek to contribute to the ongoing evolution of radiation oncology and, ultimately, to improve outcomes for cancer patients.

\section{Problematic}
Traditional manual radiotherapy planning procedures are inherently subjective and time-consuming.
The reliance on individual planner expertise often leads to variability in treatment plan quality\cite{Chung2008,Bohsung2005,Das2008,Williams2007}.
This treatment plan diversity can induce inconsistencies in patient care.
While efforts have been made to enhance consistency \cite{Bahm2011}, significant variability among planners and institutions persists.
There is a pressing need for more standardized planning approaches.
The automation of treatment planning processes presents a promising solution to these challenges.

Moreover, the time-intensive nature of manual optimization \cite{Zhang2020} and the growing demand for radiotherapy services have created a pressing need to develop automated approaches to streamline the radiotherapy planning process.
Automation enables the treatment of more patients and facilitates exploring a broader range of treatment options.

By applying computational algorithms and artificial intelligence, automated planning systems offer the potential to enhance radiotherapy treatment delivery significantly.
These systems can improve patient throughput and resource allocation by reducing planning time and increasing departmental efficiency.
Automated planning can improve plan quality and consistency, reducing variability and ensuring optimal treatment outcomes.
Furthermore, the ability to enable rapid re-planning facilitates adaptive radiotherapy, allowing for adjustments to treatment plans in response to changes in tumor volume or patient anatomy.
Finally, automated systems can explore a more expansive solution space, potentially discovering novel and innovative planning strategies that may improve treatment outcomes.

However, the development and implementation of automated planning systems pose challenges.
These challenges include the creation of robust optimization algorithms, integrating with existing Treatment Planning Systems, and validating against current clinical standards.

\section{State of the Art}
\paragraph{Automated rule implementation and reasoning}
An automated computer program with predefined rules and "if-then"\footnote{Also known as "expert" systems.} structures is a solution for implementing simple clinical guidelines.
In this context, the treatment planning system directly interprets patient anatomy and dosimetric requirements, simulating the decision-making process traditionally employed in manual treatment planning \cite{Rossille2005}.
By adhering to a structured logical framework derived from human-defined protocols, automated reasoning in radiotherapy (ARIR) can significantly reduce the need for manual intervention, particularly for repetitive tasks.

Some modern TPS vendors have incorporated ARIR solutions, offering scripting capabilities that enable users to customize the planning process.
For instance, the Varian Eclipse \cite{eclipse} system includes an application programming interface (API) that facilitates user-defined scripting functions.

\paragraph{Knowledge-based radiotherapy planning}
Knowledge-based radiotherapy planning (KBRP) represents an objective methodology for incorporating patient-specific data and historical experience into the treatment planning process \cite{Nwankwo_2014}.
By automating the optimization of KBRP, it is anticipated that a viable alternative to the current human-centric treatment planning paradigm can be established.
A prevalent KBRP approach involves leveraging a database of historical benchmark plans to learn patient-specific dosimetric parameters and generate new treatment plans.
Automated KBRP tools effectively set optimization parameters based on the desired dose-volume histogram.
Previous studies have reported notable dosimetric improvements in treatment plans generated by KBRP approaches compared to benchmark data, particularly regarding sparing organs at risk \cite{Fogliata2014,Tol2015}.

\paragraph{Conventional techniques}
Memorial Sloan Kettering Cancer Center developed advanced optimization tools, including hierarchical constrained optimization, convex approximations, and Lagrangian methods.
These tools were applied to address the complexity and enhance the speed, quality, and accessibility of a standardized, yet personalized care \cite{Zarepisheh2022}.

\paragraph{Multi Criteria Optimization}
In DVH-based inverse optimization used by most commercial TPS, a cost function must be defined to solve the minimization problem.
This function combines data from all volumes of interest as a weighted sum of penalties from each DVH constraint.
The weighting coefficients reflect trade-offs between the target(s) and different OARs.
However, this approach requires reoptimization if dosimetric preferences change during plan evaluation, making it time-consuming to find the optimal balance.
To address this, multi-criteria optimization (MCO) was introduced, enabling the simultaneous generation of multiple "anchor" plans.
Each anchor plan optimizes one OAR’s DVH criterion for maximal sparing without compromising tumor target dosimetry \cite{lahanas2003,thieke2007}.
These plans form a hypersurface in the $N$-dimension space, where $N$ is the number of independent OAR dosimetric constraints.
Referred to as the Pareto surface, this hypersurface contains the optimal plans following different dosimetric criteria.

\subparagraph{Wish list}
MCO can also be applied in an a priori approach, where dosimetric preferences are defined before inverse optimization.
In this method, a fully automated process generates a single optimal plan without the need for human interaction \cite{breedveld2007}.
Breedveld et al. introduced this approach in their work on IMRT Cycle (iCycle) \cite{iCycle}, where the optimal plan is guided by a predefined "wish list" of dosimetric criteria, each assigned a specific priority.

\subparagraph{Pareto surface narrowing}
In theory, MCO requires generating numerous plans to construct the Pareto surface, which can be time-intensive, even with automation.
However, Craft and Bortfeld analyzed head and neck IMRT plans and demonstrated that only a small number of plans are needed to form a Pareto database \cite{craft2008}.
Using objective correlation matrices and principal component analysis (PCA) of beamlet solutions, they showed that if $N$ independent dosimetric criteria are defined, $N + 1$ plans are sufficient to construct a feasible Pareto surface.
This insight facilitated the practical clinical use of MCO, first implemented in the RayStation TPS \cite{RayStationTPS}.

\subparagraph{Pareto surface exploration}
Previous research has explored Pareto optimal tradeoffs in various domains.
Gebru et al., for instance, investigated methods for evaluating Pareto optimality \cite{Gebru2023}.
Similarly, Cilla et al. conducted a comprehensive analysis of Pareto fronts \cite{Cilla2018}.
These studies provide valuable insights into the principles and techniques associated with Pareto optimization.
Craft et al. proposed a method for generating multiple Pareto optimal dose distributions, allowing clinicians to make informed decisions based on their specific preferences and clinical contexts \cite{Craft2007}

\section{Unsolved problems}
Numerous approaches have been investigated to automate the radiotherapy treatment planning process.
Despite these efforts, a consensus on the optimal method has yet to be established, and few automated solutions have been adopted in clinics.
The challenges associated with automating radiotherapy treatment planning remain a significant area of inquiry within the field.
This thesis addresses this critical issue by proposing advancements in key areas that may contribute to developing a fully automated treatment planning system.

\section{Thesis overview}
This section provides a comprehensive summary of the forthcoming chapters of this thesis.

\subsection{Fluence Map Optimization}
Chapter 3 is dedicated to developing computational methods for Fluence Map Optimization (FMO).
The chapter begins by assessing the discretization strategies.
Naive approaches for optimization are explored, stepping up to a formalization of the classical FMO problem.
A significant focus is placed on incorporating medical constraints and the corresponding importance factors associated with each constraint, which are critical in ensuring clinically viable treatment plans.
The performance of various optimization algorithms is evaluated and compared, providing insights into their relative efficiencies in solving the FMO problem.
This primary work was published in the form of an ArXiV article.

\subsection{Semi-automation}
\subsubsection*{Using graph-based analysis}
Chapter 4 explores the interrelationships between radiotherapy treatment plans by defining a distance metric.
This distance is then used to cluster plans into meaningful groups.
The clustering forms a basis for developing a semi-automated framework for treatment plan optimization.
Leveraging graph-based methodologies, this approach reduces the need for manual intervention in the planning process while ensuring the generation of high-quality plans.
The potential of this semi-automated framework to streamline clinical workflows was presented at the 2024 European Society for Radiotherapy and Oncology (ESTRO) conference.

\subsection{Full-automation}
\subsubsection*{With reinforcement learning and conventional treatment optimization techniques}
Chapter 5 addresses the limitations of classical optimization techniques in fully automating the treatment planning process.
This chapter introduces a novel framework that integrates reinforcement learning with conventional optimization.
The reinforcement learning agent is driven by clinical expertise and can be specialized to center-specific guidelines.
The findings from this work have been disseminated through a peer-reviewed journal article, a presentation at the Artificial Intelligence in Medicine (AIME) 2024 conference, and a poster presentation at the American Society for Radiation Oncology (ASTRO) 2024 meeting.

\subsection{Hybrid-automation}
\subsubsection*{Via target DVHs deep dose and dose mimicking technique}
The final research chapter, Chapter 6, presents a hybrid approach to automating treatment planning.
This approach combines deep learning for dose prediction with target Dose-Volume Histograms (DVHs) and plan-mimicking strategies.
This methodology bridges the gap between fully automated AI-driven techniques and traditional planning methods, offering a more adaptable and clinically feasible solution.
The hybrid model provides a mechanism for mimicking high-quality plans from experienced planners.
This research has been presented at the French Society of Medical Physics (SFPM) and the French Society for Radiation Oncology (SFRO) 2024 meetings.
