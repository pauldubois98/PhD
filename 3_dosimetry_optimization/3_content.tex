Biological tissues are sensible to radiations in a non-linear manner \cite{Liu2003}, and slight variations in dose can have significant biological effects.
Organs have differing sensibilities to radiation, which increases further the difficulty in formulating the goals to achieve when designing a radiation dose.
Some organs can tolerate high cumulative doses if the radiation is well distributed.
In contrast, others may withstand high doses at localized points ("hot spots") but cannot handle large doses overall.
To address these differences, clinicians impose dose-volume histogram constraints in addition to the prescribed dose.
Although the ideal objective is to minimize or eliminate radiation exposure to organs, achieving 0 Gy is impossible.
The necessity of finding compromises drives the need for advanced optimization techniques to generate fluence maps that best satisfy the medical constraints.
Therefore, various techniques can be used to calculate fluence maps (i.e., performing the critical fluence map optimization step).
In this chapter, we explore some fluence map optimization techniques.
% These techniques will be used in the remaining of the manuscrit.

\section{Naive Method}
% least squares on PTV = prescription, OARs = 0
% => does not work at all
\section{Constraints and Importance Factors}
\subsection[DVHs]{Dose Volume Histograms}
% 3D dose vs DVH
\subsection{Constraints Formulation}
\subsection{Optimization Problem}
\paragraph{Ideal}
% minimize dose while reaching constraints
\paragraph{Practical}
% get as close as possible to meeting constraints
% building of cost function

\section{Dose Mimicking}
% principle
% => is a simpler problem
% => only works if dose to mimick is nearly acheaivable
% => can arise if machine is changing during treatment
% optim problem

\section{Optimization Algorithm Review for Dosimetry}
\subsection{Introduction}
% ArXiV paper

\subsection{Methods}
%Efficiently solving this optimization problem often involves designing the objective function to be convex, thereby providing a well-defined target for the optimization process.
%Gradient-based methods, Newtonian algorithms, or quasi-Newtonian algorithms are commonly employed for this purpose.
%We aim at benchmarking state-of-the-art open-source optimization algorithms for the specific task of radiotherapy dosimetry.

\subsection{Data}
% TG-119 \cite{AAPM-TG119}
\subsection{Objective function}
% details on the one used for this study
% => tested variations (p=1,2), changed importance factors, but v similar results
\subsection{Open-source Optimizers}
\paragraph{(Stochastic) Gradient Descent}
\paragraph{Conjugate Gradient}
\paragraph{Newton}
\paragraph{SLSQP}
\paragraph{RMSprop}
\paragraph{BFGS-based}
\subparagraph{Pure BFGS}
\subparagraph{L-BFGS}
\paragraph{Adam-based}
\subparagraph{Pure Adam}
\subparagraph{RAdam}
\subparagraph{NAdam}
\subparagraph{AdamDelta}
\subparagraph{Adamax}
\paragraph{Rprop}
\paragraph{Other optimizers variations}
%In addition, we also tested AdamW, Adagrad and ASGD.
%However, AdamW \& Adagrad behaved similarly to Adam, and ASGD behaved similarly to SGD.

\subsection{Results}
% 4 TGG-119 cases figures
\paragraph{Newton's method}
\paragraph{Best Algorithms}
\paragraph{LBFGS vs BFGS}

\subsection{Discussion}
