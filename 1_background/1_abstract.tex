The background chapter of this PhD provides a comprehensive overview of key concepts in cancer treatment and radiotherapy.
If you already know about radiotherapy and multi-leaf collimator, I strongly advise to skip this chapter.

This chapter begins by outlining the nature of cancer, its phases, stages, risk factors, and common types of treatments (with their advantages and disadvantages).
Then, the physics of radiotherapy is explored, with a focus on ionizing radiation, and biological effects of radiation.
This chapter also presents the patient journey in radiotherapy, from diagnosis and treatment prescription to planning and follow-up.
Key technologies used in radiation therapy, such as multi-leaf collimator (MLC) linear accelerator (LINAC) are introduced.
Lastly, this chapter covers the irradiation techniques, and details major steps in the dosimetry process:
beam orientation optimization (BOO), fluence map optimization (FMO), leaf sequencing (LS), and direct aperture optimization (DAO).
