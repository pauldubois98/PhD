% Content of the chapter 1 of my PhD thesis
\section{Medical context}
This PhD thesis is about radiation therapy (RT) for cancer treatment.

\subsection{About cancer}
Cancer is a complex disease that can affect many parts of the body, and is a leading cause of death worldwide.
Cancer is characterized by the uncontrolled growth of cells that can invade and destroy surrounding tissues.
The World Health Organization (WHO) estimated 20 million new cancer cases in 2022, and 9.6 million deaths linked to cancer in 2022 \cite{who_cancer2022}.
Cancer touches about 20\% of the population, and is responsible for 1 in 10 deaths.

\paragraph{Cancer markers}
There are several cancer markers.
Cancer cells proliferate uncontrollably.
They also reprogram cellular metabolism to support their growth \cite{Chammas2013}.
They can also stop cell growth arrest mechanisms.
They usually manage to evade apoptosis (programmed cell death).
Cancer cells can escape the immune system, and change their cellular response phenotypic via plasticity.
At some point, cancer cells can get the ability to undergo a sufficient number of successive cell cycles of growth and division to generate macroscopic tumors.
To support their growth, they create new blood vessels to get nutrients.
Finally, they can escape and form metastasis, and will eventually provoke senescence.

\paragraph{Conditions leading to cancer}
Cancer is a complex disease.
First, cancer is caused by mutations in the DNA.
These mutations can be inherited or acquired.
Second, cancer is embraced by epigenetic reprogramming, i.e., gene expression changes (not caused by changes in the DNA sequence).
Third, cancer is often associated with an inflammatory context; inflammation can promote cancer growth and spread.
Finally, cancer is often associated with a disruption of the microbiota (the microbial community living in and on the human body).
This disruption can promote cancer growth and spread.

\paragraph{Phases of cancer}
Cancer develops in several phases.
\subparagraph{Initiation}
The first phase is initiation: a normal cell is transformed into a cancer cell.
Mutations in the DNA cause this transformation.
\subparagraph{Promotion}
The second phase is promotion or "tumorigenesis".
During this phase, the cancer cell grows and divides uncontrollably to form a tumor cluster of cells.
This growth is promoted by changes in gene expression and other factors \cite{Witsch2010}.
It may also create new blood vessels to get nutrients and oxygen.
\subparagraph{Evolution}
The final phase is evolution.
The tumor will first grow locally, then regionally, invading and damaging surrounding tissues.
Finally, the cancer cell will spread to other body parts, forming metastasis.
Metastasis is the leading cause of death in cancer patients \cite{Steeg2006}.

\subparagraph{Cancer stages}
Cancer is classified into stages \cite{cancerresearchuk2023}.
\begin{itemize}
	\item Stage 0: 'in situ neoplasm'; it means a group of abnormal cells in an area of the body. The cells may develop into cancer in the future.
	\item Stage 1: the cancer is small and contained within the organ it started in.
	\item Stage 2: the tumor is larger than in stage 1, but the cancer hasn't started to spread into the surrounding tissues.
	\item Stage 3: the cancer is larger; it has started to spread into surrounding tissues and cancer cells in the lymph nodes nearby.
	\item Stage 4: the cancer has spread from where it started to another body organ. This spread is also called secondary or metastatic cancer.
\end{itemize}
Doctors use the TNM system to describe the cancer stage \cite{Brierley2016}.
\begin{itemize}
	\item [T] stands for the size of the Tumour;
	It can be 1, 2, 3, or 4, with one being small and four being large.
	\item [N] stands for the number of lymph Nodes affected;
	It can be between 0 and 3.
	0 means no lymph node contains cancer cells; 3 means many lymph nodes contain cancer cells.
	\item [M] stands for the existence of metastasis in another part of the body.
	It can be 0 (no spread) or 1 (the cancer has spread).
\end{itemize}

\paragraph{Most common cancers}
According to the WHO, the most common cancers are lung, breast, colorectal, prostate, skin, and stomach cancer.
This thesis mainly focuses on prostate cancer, which is among the most common ones.

\paragraph{Risk factors}
Tobacco use, alcohol consumption, unhealthy diet, physical inactivity, and air pollution are risk factors for other cancer types.
However, the leading risk factor for prostate cancer is age.
Thus, it touches all social populations evenly and is unavoidable.

\subsection{Treatment types}
There are three main types of cancer treatment: surgery, radiation therapy, and chemotherapy.
The choice of treatment depends on the type and stage of cancer, the patient's age and general health, and other factors.

\subsubsection{Surgery}
Surgery is the most effective cancer treatment \cite{Klein2009}.
It involves removing the tumor and surrounding tissue.
Surgery is often used to treat early-stage cancer that has not spread to other parts of the body.
For surgery to be possible, the tumor must be located in a place the surgeon can easily access.
Surgery can be followed by other treatments, such as radiation therapy or chemotherapy, to kill any remaining cancer cells.
% give TNM / stage of cancer where surgery is possible
% give success rate of surgery

\subparagraph{Advantages}
Surgery is curative, meaning that cancer is completely removed, and the patient can "forget" about it.
It is also a local treatment, with limited side effects on the body.
Finally, only one session is needed.

\subparagraph{Disadvantages}
Surgery is invasive, and can be painful.
However, the main disadvantage, is that it can only be used for localized cancer (with no metastasis) and is accessible to the surgeon.

\subsubsection{Chemotherapy}
Chemotherapy is a treatment that uses drugs to kill cancer cells.
It is systemic, meaning it can reach cancer cells anywhere in the body.
Therefore, it usually has strong side effects.
Chemotherapy is often used to treat cancer that has spread to multiple parts of the body (i.e., metastatic cancer).

Depending on how advanced the cancer is, chemotherapy can be used to cure, control, or relieve symptoms (palliation).
% give TNM / stage of cancer where each type of chemo is possible
% give success rate as well as side effects expected

\subparagraph{Advantages}
Chemotherapy can be used to treat cancer that has spread to multiple parts of the body.
It can also be used to relieve symptoms and improve quality of life.

\subparagraph{Disadvantages}
Chemotherapy is a heavy treatment, with strong side effects.
It can also weaken the immune system, making the patient more susceptible to infections.
Finally, it can be expensive.

\subsubsection{Radiation therapy}
Radiation therapy is a treatment that uses high-energy radiation to kill cancer cells.
It is semi-local, meaning that it only affects the tumor, and the tissues traversed by the radiation beams \cite{Thariat2013}.
Radiation therapy is curative most of the time.
It can be used alone or in combination with other treatments.
% give TNM / stage of cancer where each type of RT is possible
% give success rate of RT

Radiation therapy can be delivered in two ways: external radiation therapy and internal radiation therapy.
External radiation therapy uses a machine to deliver radiation to the tumor from outside the body.
Internal radiation therapy uses radioactive materials placed directly into or near the tumor.
This thesis focuses on external radiation therapy.

\subparagraph{Advantages}
Radiation therapy is a non-invasive treatment, with limited side effects.
It is relatively localized, and can be used to treat cancers that are not accessible via surgery.

\subparagraph{Disadvantages}
Radiation therapy still targets healthy cells.
Depending on the patient's response, it may cause side effects.

\subsubsection{Other treatments}
Cancer research is very active, and new treatments are constantly being developed.
These treatments are often used in combination with others.

\paragraph{Immunotherapy}
Immunotherapy is a treatment that uses the body's immune system to fight cancer.
It can boost or change how the immune system works to find and attack cancer cells.
It is a systemic treatment.

\paragraph{Targeted therapy}
Targeted therapy is a treatment that uses drugs to target specific molecules that are involved in cancer growth.
It is a systemic treatment.

\paragraph{Hormone therapy}
Hormones are proteins or substances the body makes that help control how specific cell types work.
Hormone therapy is a treatment that uses drugs to block or lower the amount of hormones in the body that are involved in cancer growth.
It is a systemic treatment.

\paragraph{Stem cell transplant}
A stem cell transplant is a treatment that uses stem cells to replace cells damaged or destroyed by cancer treatment.
It is a systemic treatment.



% %%%%%%%%%%%%%%%%%%%%%%%%%%%%%%%%%%%%%%%%%%%%%%%%%%%% %
\section{Physics of Radiotherapy}
Radiation therapy uses high-energy radiation to kill cancer cells.

\subsection{Ionizing radiation}
Ionizing radiation has enough energy to remove tightly bound electrons from atoms, creating ions.
X-rays and gamma rays are both electromagnetic radiations that are ionizing and high-energy photons.
Some particle radiations, such as particles, beta particles, and neutrons, are also ionizing, but radiotherapy uses photon radiations.

X-rays are produced by accelerating electrons to collide with a target material and are used in medical imaging and (external) radiation therapy.
In contrast, gamma rays originate from the radioactive decay of specific atomic nuclei and are used in (internal) radiation therapy. 

Because ionizing radiation therapy can damage the DNA in cells and lead to cell death, it is used in radiation therapy for treating cancer.

\subsection{Photon interactions}
Photon-matter interactions within an absorbing medium undergo stochastic processes.
Several interactions are possible for photons.
Some random interactions generate secondary ionizing particles.
These secondary particles deposit energy in the medium.

\subsection{Photoelectric effect}
The photoelectric effect is the process by which an atom absorbs a photon, and an electron is ejected from the atom.
The photon ceases to exist, and its energy is transferred to the electron.
The ejected electron is called a photoelectron.
The photoelectron can ionize other atoms, leading to the creation of secondary electrons.
The photoelectric effect is the dominant interaction for low-energy photons.

\subsection{Compton scattering}
Compton scattering is the process by which an atom scatters a photon, and an electron is ejected from the atom.
The photon is scattered at an angle, and part of its energy is transferred to the electron.
The scattered photon is called a Compton electron.
The Compton electron can ionize other atoms, leading to the creation of secondary electrons.
Compton scattering is the dominant interaction for medium-energy photons.

\subsection{Pair production}
Pair production is when an atomic nucleus absorbs a photon and creates an electron-positron pair.
The photon ceases to exist, and its energy is transferred to the electron-positron pair.
The electron and positron can ionize other atoms, leading to the creation of secondary electrons.
Pair production is the dominant interaction for high-energy photons.

% add graphic of photon interactions
% Pair production, the photoelectric effect, and Compton scattering
% constitute the primary interaction mechanisms observed in the megavoltage energy range.

\subsection{Photon attenuation}
The photon beam will be attenuated as it passes through the medium, and its intensity will decrease.
The dose deposition in the medium is proportional to the intensity of the photon beam.
The attenuation of the beam follows an exponential law concerning the depth of the medium traversed (Lambert-Beer law) \cite{Beer1852}:
$$I(x) = I_0 \exp(-\mu x)$$
where $I$ is the intensity of the photon beam after passing through a thickness $x$ of the medium,
$I_0$ is the initial intensity of the photon beam,
and $\mu$ is the attenuation coefficient of the medium.

% \paragraph{Non-ionizing radiation}
% Non-ionizing radiation is radiation that does not have enough energy to remove tightly bound electrons from atoms.
% This type of radiation is not used in radiation therapy.

% %%%%%%%%%%%%%%%%%%%%%%%%%%%%%%%%%%%%%%%%%%%%%%%%%%%% %
\section{Biological effect on cells}
Ionizing radiation can damage the cells, leading to cell death in various ways.

\subsection{Radiation effects on DNA}
Ionizing radiation damages the DNA \cite{Scholes2014} in cells and leads to cell apoptosis, necrosis, or senescence.
Radiation induces DNA damage through both direct and indirect mechanisms:
Directly, it causes single-strand breaks (SSBs), double-strand breaks (DSBs) \cite{Roots1985}, DNA crosslinks, and DNA-protein crosslinks \cite{Nakano2017}.
Indirectly, radiation generates reactive oxygen species (ROS) and reactive nitrogen species (RNS), further contributing to DNA damage.

\paragraph{DNA repair}
Cells have mechanisms to repair DNA damage.
There are several types of DNA repair mechanisms, including base excision repair (BER), nucleotide excision repair (NER), mismatch repair (MMR), and double-strand break repair (DSBR).
Cancer cells often have defects in DNA repair mechanisms, making them more sensitive to radiation therapy \cite{Brierley2016}.
% DNA quality plot with DNA reparation only for healthy cells

\subsection{Radiation affects the plasma membrane}
Radiation significantly impacts the biological properties of the plasma membrane by affecting its composition, structural integrity, and functional capabilities.
Radiation exposure can alter the fluidity and permeability of the cell membrane, affecting the transport of ions and molecules into and out of the cell.
Additionally, radiation causes corrosive damage, and damage to the membrane can initiate signaling events that are important for the apoptotic response \cite{CohenJonathan1999}.
These changes can have cascading effects on various cellular processes, highlighting the critical role of the plasma membrane in maintaining cellular homeostasis under stress conditions.

\subsection{Radiations and cell organelles performances}
Radiation exerts significant detrimental effects on various cellular organelles, impacting their functionality and overall cellular health \cite{Somosy2000}.
One critical target of radiation damage is the endoplasmic reticulum, where radiation can disrupt protein folding and processing, leading to cellular stress and apoptosis.
Additionally, ionizing radiation induces alterations in ribosomal structure and function, impairing protein synthesis and compromising cellular homeostasis.
Mitochondria, the cell's powerhouses, also exhibit altered behavior following radiation exposure, including disruptions in energy production and initiating apoptotic pathways.
Furthermore, lysosomes, essential for cellular waste processing and recycling, suffer damage upon irradiation, potentially accumulating cellular debris and impairing cell function.
These collective effects highlight radiation's broad and profound impact on cellular organelle performance \cite{Wang2018}.

\subsection{Radiation alters the biological behavior of tumor cells and the immune system}
Radiation profoundly influences the biological behavior of tumor cells and the immune system, impacting critical aspects of cancer progression and immune response.
It affects tumor cell proliferation, often reducing the ability of cancer cells to multiply by damaging their DNA and cellular structures.
Radiation also influences tumor cells' invasion and metastasis potential, either by directly impairing their motility or altering the tumor microenvironment to make it less conducive to cancer spread.
Additionally, radiation can modulate cancer-promoting inflammation, either by inducing pro-inflammatory signals that support tumor growth or by disrupting the inflammatory milieu to hinder cancer progression.

\subsection{Radiation effects when combined with immunotherapy}
Radiation may support immunotherapy, making the effect of both treatments more significant than the sum of their impact if used alone.

\paragraph{Ray-Enhanced Anti-CTLA-4 Immunotherapy}
Radiation therapy can enhance the efficacy of anti-CTLA-4 immunotherapy, a treatment that blocks the CTLA-4 protein in T cells, thus boosting the immune system's response against cancer cells.
The combination of radiation and anti-CTLA-4 immunotherapy has shown promising results, as radiation-induced tumor cell death releases antigens that can further stimulate the immune system \cite{Vanpouillebox2015}.
This synergy can improve tumor control and potentially better clinical outcomes than either treatment alone.

\paragraph{Radiation Combined with Anti–PD-1/PD-L1 Immunotherapy}
Combining radiation with anti–PD-1/PD-L1 immunotherapy has shown significant success.
Anti–PD-1/PD-L1 therapies block the PD-1/PD-L1 pathway, which tumors exploit to evade immune detection.
Radiation therapy can augment this effect by increasing the immunogenicity of the tumor, thereby making cancer cells more susceptible to immune attack \cite{He2021}.

\paragraph{TLR-Mediated Immunologic Effects of Radiation Therapy}
Radiation therapy can also exert immunologic effects through Toll-like receptors (TLRs), a class of proteins involved in pathogen recognition and activation of innate immunity.
Radiation can activate TLRs on immune cells, producing cytokines and chemokines that enhance the immune response against tumors.
This TLR-mediated effect contributes to the synergy between radiation and immunotherapies, leading to more robust anti-tumor responses \cite{Walshan2020}.

While this thesis does not focus on biological aspects, one should remember that radiation affects cells in various ways, and cancer therapy is complex.

% %%%%%%%%%%%%%%%%%%%%%%%%%%%%%%%%%%%%%%%%%%%%%%%%%%%% %
\section{Patient Path}
The radiotherapy patient path encompasses several critical stages, each essential for the effective treatment of cancer.
This section outlines the sequential steps of radiotherapy, from initial detection and diagnosis to follow-up care.

% add a graphic
\subsection{Diagnostic}
Patients diagnosed with a tumor can go through several paths: surgery, radiotherapy, immunotherapy, chemotherapy, or any combination.
Doctors will choose the most appropriate treatment(s) based on evidence they have (biopsy, radios, et cetera).
This manuscript will focus on the radiotherapy path.

\subsection{Radiotherapy Prescription}
Following a confirmed diagnosis and the choice of radiotherapy treatment, the oncologist develops a prescription.
This prescription specifies the type, dosage, and frequency of radiation treatment tailored to the patient's specific cancer type, location, and stage.
The doctors define minimal tumor irradiation and maximum damage to surrounding healthy tissues.
Most of the time, templates are used and fine-tuned to fit specific patients.

\subsection{CT scan and Contouring}
A computed tomography (CT) scan is performed to obtain detailed images of the patient's anatomy.
These images are used to delineate the tumor and surrounding organs at risk.
The contouring task used to be a manual operation but is now done automatically, thanks to the progress of artificial intelligence on segmentation tasks.
The CT scan also provides the spatial information necessary for precise irradiation simulation.

\subsection{Treatment Planning}
The treatment planning process involves developing a detailed plan specifying the patient's radiation dose distribution.
Advanced software calculates the optimal arrangement of radiation beams to achieve the desired dose while minimizing exposure to healthy tissues.
This thesis registers new advances in the planning step.
Plans must be reviewed and approved by doctors.

\subsection{Irradiation Sessions}
Irradiation sessions, or treatment delivery, is the actual irradiation of the patient.
Cone Beam Computed Tomography (CBCT) is usually done to reposition the patient with the scan so that all organs align with the planning CT.
Nowadays, the tendency is to reduce the number of irradiation sessions (the old typical five weeks of five sessions is now usually two weeks of five sessions).

\subsection{Follow-up}
After the completion of radiotherapy, patients enter the follow-up phase.
Regular follow-up appointments are scheduled to monitor the patient's response to treatment, manage any side effects, and detect any signs of recurrence.

\section{Machines}
The discovery of X-rays by German physicist W. C. Roentgen in 1895 marked a pivotal moment in medical science.
Only one year later, in 1896, Despeignes began using radiotherapy in France.
Victor Despeignes delivered 15-30 minutes with 80 irradiation sessions (so-called "fractions") to relieve the pain of patients with stomach cancer \cite{Holsti1995}.

Since then, machines have become more powerful and more complex.
Modern machines can deliver mega-voltage radiation \cite{Huh2020}, which are sufficiently high to destroy tumors in minutes.
However, such high-power treatments will irreversibly damage healthy tissues.
Hence, as the machines became more powerful, constructors built more complex modulation mechanisms to preserve organs at risk.

% add graphic of usage per machine type, and per constructor
\subsection{Molds}
The first kind of modulation used was molds: their purpose was to stop the irradiation before it reached the body.
By strategically obscuring the rays, organs can be spared, as they will receive a small amount of irradiation dose, while the tumor will receive a fatal dose.
Molds had significant limitations due to their single-use nature.
It was necessary to create a custom mold for each patient as their anatomy differs.
Typically, three molds were required for the three irradiation angles.
Modern technology avoids single-use molds using motorized blockers to stop the rays and dynamically modulate the radiation beam.

\subsection[MLC-LINAC]{Multi-Leaf Collimator - LINear ACcelerators}
Multi-Leaf Collimator (MLC) technology combined with Linear Accelerators (LINAC) was a revolution in the radiotherapy world \cite{Bakiu2013} \cite{Xu2017}.
In MLCs, an array of motorized leaf pairs can shape the radiation beam with high precision.
% add picture + schematics
Additionally, MLC systems are sometimes equipped with jaws, which help to shape the beams better.
The MLC-LINAC is the most common type of radiation therapy machine used today.
This manuscript will focus on the MLC-LINAC system due to its widespread use and versatility in clinical settings.

\subsection{Tomotherapy}
Tomotherapy systems have an irradiation head that rapidly rotates around the patient, equipped with a single layer of binary blockers that can be activated and deactivated almost instantaneously \cite{Mackie1999}.
The tomotherapy machines follow a helical path \cite{Jeraj2004}, rotating around the patient while simultaneously moving along their body.

\subsection{CyberKnife}
CyberKnife systems are another non-invasive alternative to conventional MLC-LINAC radiotherapy machines with higher flexibility \cite{Kilby2020}.
These CyberKnife machines have the irradiation head mounted on a robotic arm, which allows a vast array of motions around the patient.
This flexibility enables the delivery of even more complex-shaped doses.
CyberKnife technology is particularly beneficial for treating unusual tumors in challenging or sensitive body areas.

\subsection{Brachytherapy}
Brachytherapy involves the placement of a radioactive source directly inside the body of the patient \cite{Chargari2019}.
This technique allows for delivering localized high-dose radiation.
Although brachytherapy involves an invasive procedure, it significantly minimizes radiation exposure to surrounding healthy tissues.
The localized character of brachytherapy makes it a good treatment option for some types of cancer.

\section{Irradiations techniques}
This section describes the main irradiation techniques that can be used with MLC-LINAC machines.
The techniques have evolved over the years of MLC usage.
Better irradiation techniques improve tumor targeting while keeping exposure of healthy tissues to a minimum.

\subsection[3D-CRT]{3-Dimensional Conformal Radiotherapy}
Three-Dimensional Conformal Radiotherapy (3D CRT) shapes radiation beams to closely fit the contours of the tumor.
The MLC leaves are positioned to match the tumor's contour projection on a plane perpendicular to the radiation rays, typically using three angles.
Such shaping of the beams can be done with mold (which is single-use) or with an MLC.
Although 3D CRT targets the Principal Target Volume (PTV) more than Organs At Risk (OARs), modern techniques provide superior sparing of healthy tissues.
Consequently, advanced and less naive methods have largely supplanted 3D CRT in contemporary clinical practice.

\subsection[IMRT]{Intensity Modulated RadioTherapy}
Intensity Modulated RadioTherapy (IMRT) represents a significant advancement over 3D CRT by taking better advantage of the  MLC capabilities.
Instead of delivering radiation with uniform intensity from each angle, IMRT dynamically adjusts the beam intensity to improve patient outcomes \cite{Tubiana2000}.

\paragraph[S\&S]{Step and Shoot}
The "Step and Shoot" technique in IMRT involves sequentially moving the MLC leaves to different positions to deliver varying radiation intensities.
This method enables the delivery of fluence maps that resemble level sets. 
Before delivery, the fluence maps must be discretized to level sets of nearly convex shapes.

% add illustration

\paragraph[SW]{Sliding Window}
The "Sliding Window" technique employs a continuous sweep motion of the MLC leaves, determined by solving a linear programming problem.
This approach theoretically enables the delivery of any continuous, positively defined fluence within the irradiation window of the MLC-LINAC.
The irradiation time is reduced compared to step and shoot.
This manuscript assumes the use of the Sliding Window technique, focusing on optimizing the fluence distribution.

% add details?
% add illustration
% add link to web tool

\paragraph{Number of Beams}
The choice of the number of beams in IMRT is a balance between treatment complexity and effectiveness.
Using many beams can evenly spread the unwanted dose across all organs, but adds complexity to treatment planning and prolongs the delivery time, which can increase patient movement and reduce dose precision.
Conversely, fewer beams simplify planning and shorten treatment time but may result in less optimal dose distribution.
Research indicates that 50 beams are needed for "nearly optimal IMRT" \cite{Fenwick2011}.
Beams at exactly 180 degrees from each other tend to have (very) similar influence on the dose distribution on the patient.
Therefore, dosimetrists tend to choose an odd number of equispaced beams.
In practice, the number of beams used is below 25.

% add illustration of 2 vs 4
% add illustration of 3,5,7,9,11 beams

\subsection[VMAT]{Volumetric Modulated ArcTherapy}
Volumetric Modulated Arc Therapy (VMAT) enhances IMRT by allowing the MLC LINAC head to rotate while delivering radiation.
Unlike IMRT, which stops the head at specific positions around the patient, VMAT continuously irradiates while rotating.
This technique can better distribute the unwanted dose and reduces the irradiation time \cite{Hardcastle2011}.

% add illustration of IMRT-5 vs VMAT

However, the mechanical constraints of the machine complicate the optimization problem for VMAT compared to IMRT, making the optimization more computationally intensive.
Studies have demonstrated that, IMRT with a Sliding Window and more than 7 angles can achieve equally effective dose distribution \cite{Bortfeld2010} \cite{Quan2012}.
While demonstrated with IMRT Sliding Window, the techniques developed in this manuscript apply to VMAT, given sufficient computational resources.


\section{Treatment Planning Systems}
Treatment Planning Systems (TPS) are the crucial tools that calculate the precise machine (MLC) motions according to the dosimetrists priorities and the irradiation technique chosen.

\subsection{Manufacturer}
\paragraph{Eclipse\texttrademark\ (Varian)}
Eclipse\texttrademark\ \cite{eclipse}, developed by Varian, is one of the most widely used TPS globally.
It supports VMAT with one or multiple arcs, and IMRT with any number of beams.
Eclipse\texttrademark\ integrates with Varian's suite of treatment machines, and integrates an automatic contouring tool \cite{eclipse_brochure}.

\paragraph{ONE® | Planning (Elekta)}
ONE® | Planning \cite{one_planning} is Elekta's TPS, and is also widely used, supporting IMRT and VMAT.
It is renowned for its speed with high-precision dose calculation using the Monte Carlo method.

\paragraph{Precision® (Accuray)}
Developed by Accuray, Precision® \cite{precision} is the dedicated TPS for CyberKnife and TomoTherapy systems.

\subsection{Non-manufacturer}
\paragraph{RayStation (RaySearch)}
RayStation, developed by RaySearch Laboratories, is a TPS known for its advanced optimization algorithms.
Unlike manufacturer-specific systems, RayStation can output plans for a wide range of linear accelerators and imaging devices.
It offers robust support for various treatment techniques, including VMAT, IMRT, 3D-CRT, Cyberknife, and TomoTherapy.

\paragraph{matRad (German Cancer Research Center - DKFZ)}
matRad is an open-source TPS developed by the German Cancer Research Center (DKFZ) for research and education.
While not intended for clinical use, matRad offers a flexible platform for testing and developing new treatment-planning algorithms.

\paragraph{AutoPlan (TheraPanacea - Unpublished)}
AutoPlan is the upcoming TPS from TheraPanacea, designed to incorporate artificial intelligence and machine learning into the treatment planning process.

\section{Dosimetry steps}
\paragraph{Challenges}
% competeing goals
\subsection[BOO]{Beams Orientation Optimization}
% boo is useless \cite{Rocha2019}
\subsection[FMO]{Fluence Map Optimization}
\subsection[LS]{Leaf Sequencing}

\section{Simulation}
% https://oncologymedicalphysics.com/dose-calculation-algorithms/
