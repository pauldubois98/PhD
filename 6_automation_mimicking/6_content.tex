% https://github.com/pauldubois98/SFPM-JS2024
\section[DVH Guided Deep Dose]{Dose-Volume Histograms Guided Deep Dose Prediction for Radiotherapy Treatment Planning (SFPM 2024)}
\subsection{Introduction}
Traditionally, the creation of radiotherapy treatment plans has been a semi-manual process, where dosimetrists finetune importance factors assigned to structures and constraints.
A cost function is then used through a classical optimization algorithm to calculate the optimal plan.

In recent years, deep learning in treatment planning has gained attention.
Deep learning models can predict the three-dimensional dose distribution based on patient-specific anatomical data derived from medical imaging (CT scans).
While the predicted dose distribution may not directly represent a deliverable treatment plan, it serves as the basis for determining a clinically viable plan through dose mimicking.
Dose mimicking is an optimization technique that eliminates the need for manual adjustment of importance factors by dosimetrists.
Therefore, the ability to predict a clinically acceptable and near-deliverable 3D dose distribution for any patient presents significant potential for fully automating the radiotherapy planning process.
It is important to note that the successful application of dose mimicking requires a target dose distribution that is nearly deliverable; thus, arbitrarily setting the target dose to zero for OARs is not feasible.

However, this approach requires further adaptation to accommodate specific clinical guidelines.
A potential solution involves training individualized models for each treatment center, allowing institution-specific practices and guidelines to be incorporated.
However, deep learning dose prediction models are computationally large, and implementing separate models for each center is resource-intensive.
Furthermore, such models require substantial datasets for effective training.
Consequently, smaller treatment centers may lack the necessary data volume to train a comprehensive model adequately.
Additionally, a separate model may be required for each prescription type due to the variability in prescription doses, making manual treatment planning necessary for non-standard cases.
Finally, clinicians and dosimetrists may prefer manually adjusting treatment plans in some cases.
Such adjustments are not feasible within the current model framework.

We propose a novel approach that incorporates target Dose-Volume Histograms (DVHs) directly into the input of the deep learning-based dose prediction model.
Incorporating DVHs introduces interactivity into the model, allowing adjustments to the target DVH to yield corresponding changes in dose predictions.
This methodology enables a workflow where dosimetrists can refine the predicted dose distribution according to specific clinical objectives.
Furthermore, by establishing a template target DVH tailored to each clinic, the same model can be deployed across multiple centers while generating 3D dose predictions that align with the specific practices of each institution.

\subsection{Material and Methods}
%We used a cohort of 168 patients from the ICM radiotherapy department (Institut régional du Cancer de Montpellier).
%We have developed a deep-learning model to predict the 3D dose distribution for radiotherapy treatment planning.
%We used a standard U-net with the CT scan, the Principal Target Volume (PTV) contour, the Rectum contour, and the Bladder contour as inputs.
%To incorporate DVH information, the model uses a technique called Direct Affine Feature Transforms (DAFT)2.
%Three models were compared: one without DVH data (model C, classical U-net), one with DVH data incorporated at the bottleneck layer (model B), and one with DVH data incorporated throughout all encoder-decoder connections (model A).

\subsection{Results}
%Quantitatively, the networks incorporating the DVH data performed better, with Mean Absolute Error (MAE) on a test set of 2.42 Gy (model A), 2.58 Gy (model B), and 3.18 Gy (model C).
%The dataset comprises patients prescribed 62 Gy on the PTV and others 78 Gy on the PTV.
%We observe that model C could not adapt to the different prescriptions and consistently predicted a dose that looks like a 65 Gy prescription.
%Conversely, models A and B adjusted their deep dose predictions to the prescription.
%These modifications in the predicted 3D dose indicate that the models adapted their predictions to fit the DVHs.

\subsection{Conclusions}
%Our study demonstrates the possibility of incorporating DVH data into deep dose generation models.
%Dose prediction is more accurate with DVHs, and our model adapts better to varying prescriptions.
%This technique allows a new dose optimization workflow where dosimetrists only need to design the DVHs that suit them.
%The TPS will compute the deliverable plan that best matches the DVHs asked.
%Moreover, while one cannot transfer a 3D dose from one patient to another, DVHs are comparable across patients.
%Hence, after finding one DVH set that suits a center’s practices, calculating the optimal plan for new patients will only need minor modifications.

%References
%1.	Chris McIntosh, Mattea Welch, Andrea McNiven, David A Jaffray, and Thomas G Purdie.
%Fully automated treatment planning for head and neck radiotherapy using a voxel-based dose prediction and dose mimicking method.
%Physics in Medicine Biology.
%doi: 10.1088/1361-6560/aa71f8
%2.	Sebastian Pölsterl, Tom Nuno Wolf, and Christian Wachinger.
%Combining 3D Image and Tabular Data via the Dynamic Affine Feature Map Transform, page 688–698.
%Springer International Publishing, 2021.
%ISBN 9783030872403.
%doi: 10.1007/978-3-030-87240-3 66



%%%%%%%%%%%%%%%%%%%%%%%%%%%%%%%%%%%%%%%%%%%%%%%%%%%%%%%%%%%%%%%%%%%%%%%%
%                                                                      %
%   %%%%%%%%%%%%%%%%%%%%%%%%%%%%%%%%%%%%%%%%%%%%%%%%%%%%%%%%%%%%%%%%   %
%   %%%%%%%%%%%%%%%%%%%%%%%%%%%%%%%%%%%%%%%%%%%%%%%%%%%%%%%%%%%%%%%%   %
%   %%%%%%%%%%%%%%%%%%%%%%%%%%%%%%%%%%%%%%%%%%%%%%%%%%%%%%%%%%%%%%%%   %
%   %%%%%%%%%%%%%%%%%%%%%%%%%%%%%%%%%%%%%%%%%%%%%%%%%%%%%%%%%%%%%%%%   %
%                                                                      %
%%%%%%%%%%%%%%%%%%%%%%%%%%%%%%%%%%%%%%%%%%%%%%%%%%%%%%%%%%%%%%%%%%%%%%%%


% https://github.com/pauldubois98/SFRO2024
\section{Attention Mechanism on Dose-Volume Histograms for Deep Dose Predictions (SFRO 2024)}
\subsection{Introduction}
%Radiotherapy treatment planning aims to deliver a targeted radiation dose while minimizing damage to healthy tissues.
%Traditionally, dosimetrists define constraints for dose delivery through a semi-manual process.
%Deep learning techniques have emerged to automate parts of this planning by predicting the 3D dose distribution based on patient scans, and then use “dose mimicking” techniques to compute the plan.
%However, these techniques lack interactive feedback for dosimetrists within the treatment planning system (TPS).
%This study proposes using attention mechanism to incorporate target Dose-Volume Histograms (DVHs) into the dose prediction model.
%This enables interactive modifications of the plan, by modifying the target DVHs.

\subsection{Material and Methods}
%We used a cohort of 168 patients (with split 80-10-10 for training-validation-test).
%The models input CT scans, Principal Target Volume (PTV) contour, and Organs at Risk (OARs ) contours; and output a 3D dose.
%We trained the models with a voxel-wise Mean Absolute Error (MAE) , combined with a DVH L1 loss.
%The backbone of the models is a convolutional Unet of depth three.
%Then, we compared three architectures.
%First, a vanilla Unet-0 (with no DVH information).
%Second, a Unet-1 with DVH information incorporated using the DAFT technique.
%Third, a Unet-2, with DVH added via a cross-attention mechanism (our contribution).
%The cross attention takes queries from the 3D data, and keys/values from the target DVHs.
%Finally, the attention output is re-injected in the Unet-3 model via a simple addition.


\subsection{Results}
%Models incorporating DVH data (Unet-1 and Unet-2) achieved superior performance.
%The MAE and mean DVH deviation were improved by Unet-1 and Unet-2, with a minor advantage for the Unet-2.
%
%%Performances of the three models
%%Metric	Unet-0	Unet-1	Unet-2
%%3D dose MAE	3.093 Gy	2.254 Gy	2.210 Gy
%%Mean DVH deviation	1.942 Gy	1.051 Gy	0.930 Gy
%
%The dataset included patients prescribed for either 62 Gy or 78 Gy on the PTV.
%The Unet-1 model was not able to adapt to varying prescriptions, consistently predicting a dose resembling a 65 Gy prescription.
%Conversely, Unet-1 and Unet-2 adjusted their dose predictions suggesting that these models adapted their predictions to be conform with the provided DVHs.

\subsection{Conclusions}
%This study demonstrates the feasibility of incorporating DVH data into deep learning models for dose prediction.
%The inclusion of DVHs resulted in improved dose prediction accuracy and enhanced model adaptability to varying prescriptions.
%This approach paves the way for a novel dose optimization workflow where dosimetrists primarily focus on designing desired DVHs.
%The TPS would then compute the deliverable plan that best matches the specified DVHs.
%Furthermore, DVHs offer greater inter-patient comparability compared to 3D dose distributions.
%This enables the establishment of a standardized target DVH per treatment center.
