% 1 - INTRO
\chapter*{Acknowledgments}
During those three years, I often felt alone, but I eventually realized that I could not name people who supported me, not because there were too few, but because there were so many.
This section is dedicated to all those who directly or indirectly contributed to this manuscript.



% 2 - PHD DIRECTORS
I want to express my deepest gratitude to my PhD directors for their unwavering support and invaluable guidance throughout this journey.

To Nikos Paragios, thank you for the network and opportunities you have opened for me.
Your insight and encouragement have broadened my horizons and enriched my academic experience.

To Pascal Fenoglietto, your guidance has been indispensable.
Your thoughtful advice and attention to detail have steered me through the complexities of this work, and I sincerely appreciate your dedication to my progress.

To Paul-Henry Cournède, you have been more than a supervisor; you are a model of excellence in research and leadership.
I am immensely grateful for inspiring me to reach new heights.



% 3 - THERAPANACEA
I would like to thank the TheraPanacea team, these numerous individuals who have contributed to my PhD journey in various ways:

To Alexandre Cafaro, for our exchanges on research.
Sharing our experiences has been a source of motivation.

To Alexis Benichoux, for valuable insights into AI engineering practices.

To Alexis Bombezin-Domino, for sharing interesting facts e.g.: about military aviation.

To Amaury Leroy, for insightful discussions about PhD life and post-PhD life plans.

To Anne Walrafen, for sharing extensive knowledge about radiotherapy.

To Anothkanth Mahendran, for maintaining and updating specialized software, essential to my research.

To Audrey Duran, for maintaining nice colleague relationship despite the physical distance.

To Ayoub Oumani, for showing great skills on the soccer field during our 5v5 after-work matches.

To Baris Ungun, for fascinating discussions about unconventional scientific topics, like color theory.

To Basile Bertrand, for making ESTRO conference an enjoyable and memorable experience.
Should I express my gratitude for sending me to ASTRO instead of attending yourself?

To Carlos Santos-Garcia, for being a true friend across multiple domains - from climbing to soccer, from AI discussions to board games.

To Catherine Martineau-Huynh, for showing genuine care for employees, and keeping us on our toes with unexpected questions.

To Claire Diaz, for bringing joy and enthusiasm to after-works and conferences.

To Clemence Gueguen, for the privilege of knowing you as both student and colleague.
Many of us thank you for organizing memorable company events.

To Despina Ioannidou, for managing internal \xcancel{gossips} communications with grace and discretion.

To Edouard Delasalles, for enlightening discussions about AI techniques.

To Elie Mangin, for leading me at the Centrale Night'N'Day race, and being a worthy baby-football opponent, and a great colleague.

To Ethan Corcos, for engaging scientific exchanges, and showing me cool figures.

To Eugenie Ullmann, for being a great colleague.

To Giorgi Benashvili, for resolving all my technical issues promptly and efficiently.

To Gizem Temiz, for the gentle push toward abstract submissions, contributing to my academic growth.

To Jacob Buatti, for being an outstanding volleyball player and bringing athletic spirit to TheraPanacea.

To Jules Potel, for enriching scientific discussions.

To Lhassa Macke, for bringing energy and liveliness to the clinical room.

To Lisa Letournel, for sharing Catherine's front desk office, and maintaining a nice atmosphere.

To Lorenzo Colombo, for animating the clinical office room with complains, and successfully organizing the last company event.

To Louis Ducamp, for enlightening discussions about AI applications in healthcare.

To Léo Hardi, for patiently answering my database questions, no matter how basic they seemed.

To Mathilde Ravier, for meaningful conversations about company life and being a reliable running companion.

To Niels Pichon, for productive exchanges about development of AI tools.

To Norbert Bus, for an inspiring performance at Advent of Code.

To Olivier Teboul, for exemplary leadership of the scientific team.

To Quentin Spinat, for being an excellent climbing partner, providing much-needed breaks from work.

To Rafael Marini Silva, for providing invaluable guidance.

To Rafael Roblin, for engaging technology discussions and creative problem-solving during team building events.

To Rémi Vauclin, for stimulating scientific discussions.

To Sami Romdhani, for leading the AI team, and letting me participate to interesting meetings.

To Samia Achour, for expertize coordinating conferences, facilitating valuable networking.

To Sanmady Kandiban, for being a great colleague.

To Sofia Broome, for showing genuine interest in my work, and always having interesting questions.

To Sofiane Horache, for discussions on AI that helped shape my research.

To Sonia Martinot, for being an excellent DSBA co-teacher and sharing PhD life experiences.

To Tessa Kolb, for patiently guiding me through ASTRO demonstrations and enhancing my technical understanding.

To Thaís Roque, for bringing a bit of Oxford experience to TheraPanacea.

To Vincent Luc, for making SFPM conference a valuable experience.



% 4 - MICS
I would like to extend my heartfelt gratitude to my colleagues and friends from the MICS laboratory, whose presence and support have enriched the past three years in countless ways:

To Aaron Mamann, for our endless and enriching life discussions that often provided unneeded breaks, but relieved from academic pressures.

% To Agathe, who officially welcomed me into the MICS family by adding me to the laboratory's webpage.

To Antonin Della Noce, for our engaging discussions about mathematics that challenged my thinking.

To Brice Hannebicque, for our enlightening discussions about teaching strategies.

To Céline Hudelot, thank you for trusting me to be your teaching assistant for computer vision and for advertising my enigmas, which has contributed to making them part of our lab culture.

To Fabienne Brosse, for expertly managing the administrative aspects of the lab and (most notably) for organizing our summer and Christmas parties.

To Félicie Giraud-Sauveur, for being my desk mate and creating an enjoyable environment.

To Gabriel Claret, for his incredible devotion to the lab life.

To Guillaume Joslin, for helping out with the computing resources.
I do \textbf{not} thank you for hacking my enigmas!

To Gurvan Hermange, whose perspectives on PhD experience have helped me better understand the academic ecosystem.

To Hakim Benkirane, for our discussions about the latest developments in AI.

To Imane Chraki, for her contribution to lab life and the fun times we shared at the autumn school in Porto.

To Inès Malleval, for creating a laboratory sportive life.

To Jun Zhu, whose exemplary work ethic has been truly inspiring.
Thank you for always keeping me company during lunch!

To Konstantinos Florakis, for your endless enthusiasm for mathematical discussions.

To Laura Vuduc, for being a wonderful companion on our bus rides to CentraleSupélec, and for being an exceptional colleague throughout our PhD journey.

To Leo Filioux, for our enlightening discussions about teaching.

To Leo Milecky, for your sage advice on publication strategy and those memorable spike ball victories.

To Lily Monnier, for collaborating with me in teaching reinforcement learning, for being part of the autumn school in Porto team, and for our mutual encouragements during the PhD.

To Mahmoud Bentriou, for sharing your expertise on publication.

To Malek Ben Salah, for being a central member of the biomathematics team.

To Maria Vakalopoulou, for organizing the Biomath seminars, which have been an incredible source of inspiration and knowledge.

To Marin Scalbert, for your intermittent yet enjoyable appearances at the lab.

To Marine Tesson, for her support with high school internship.
It is always a pleasure to work with you.

% To Myriam Tami, for always encouraging\footnote{unfortunately failing} us to take the initiative and organize seminars.

To Othmane Laousy, for his insights about reinforcement learning (among others).

To Quentin Blampey, for his contribution to the lab, and for challenging during the climbing after-works.

To Romain Lhotte, for consistently challenging my coding and pushing me to excel.
Teaching courses together has been an invaluable learning experience.

To Stergios Christodoulidis, for allowing me to teach reinforcement learning.
His involvement in the autumn school in Porto have greatly enriched my academic journey.

To Sylvain Lannuzel, for his perspectives on PhD life and willingness to share his experience.

To Vessna Lukic, for our varied life discussions.
Your ability to balance research with life's other aspects is genuinely inspiring.

To Vincent Mousseau, for validating my "heures de formation"\footnote{even oenology was accepted}.

To Véronique Letort, thank you for providing teaching opportunities, and valuable advice.

To Wassila Ouerdane, for providing the opportunity to teach NLP.

To Yoann Pradat, for our stimulating discussions about science and the intricacies of academic publishing.



% 5 - OTHER PHD CANDIDATES
Beyond the walls of MICS and Therapanacea, I've been fortunate to share this doctoral journey with fellow PhD candidates and friends.
I would like to express my gratitude to the following individuals for their friendship, support, and challenges:

To Aiden Manley, for being not just a flatmate but a true friend, making our shared living space a home during this academic journey.

To Amaury Ajasse, for being  a great partner, both in sports and theater.

To Arthur Vervaet, for your unwavering friendship, for consistently pushing my limits at the climbing gym, and for our thoughtful discussions about career trajectories and aspirations.

To Axel Kerbec, for our stimulating mathematical discussions and for our mutual drive to challenge each other intellectually, pushing us both to grow and excel.

To Camille Béhar, for your friendship and for the humbling reminder that unlike my path in computer science, you will become a "real" medical doctor, bringing perspective to our different journeys in academia.

To Caroline Bouat, for maintaining our friendship through the years, and pursuing together a scientific PhD.

To Jeanne Redaud, for being an exceptional friend and confidante, sharing the unique challenges and triumphs of PhD life at L2S, making this journey feel less solitary.

To Landry Duguet, for our enduring friendship and the remarkable parallel journey we've shared in our scientific careers.

To Marie Girodengo, for sharing the unique experiences and concerns of PhD life, making this challenging journey more manageable.

To Pierre Houdouin, for our engaging climbing sessions that provided much-needed breaks from research, and for our meaningful discussions about PhD progress and milestones at L2S.



% 6 - OTHER FRIENDS
I've been fortunate to have friends outside academia whose insights, support, and contributions have enriched both this manuscript and my understanding of its broader impact.

To Axel Arno, for the hours we spent talking about mathematics, and creating/testing problems.

To Enea Sharxhi, for the countless hours spent mutually challenging each other with mathematical problems, often pushing the limits of our understanding and problem-solving abilities.

To Erwan Le Guennec, for sharing great climbing sessions.

To Julien Bruyninckx, for ensuring I maintained a balanced life beyond science while engaging in fascinating discussions about mathematics, quantum computing, and AI.

To Justin Cuzin, for our laughs and tears while teaching together.

To Louis Lhotte, for our engaging discussions about career, and being a great employee.

To Maxime Dufour, for forming an extraordinary hackathon team with Romain and Arthur, where our combined creativity led to innovative solutions and memorable coding adventures.

To Mélanie Kojaartinian, for her careful review and insightful feedback on the background chapter, contributing to its clarity and accessibility.

To Oscar Valdini, for our thoughtful discussions about career paths and possibilities.

To Roderick Rens, for being one of my longest-standing mathematician friend.

To Thomas Heyen-Dube, for sharing invaluable life wisdom and unwavering support.



% 7 - FAMILY
Despite already having several doctors in our family tree, I somehow found myself trying to add another branch to this academic legacy.
This journey would not have been possible without the unwavering support of my family, who have been my foundation throughout this adventure.

To my grandparents (Jean-Bernard, Marie-Paule, Maxime, Maguy), whose pride in their grandchildren's achievements has always been a source of motivation.

To my uncles (François, Philippe, Sylvain, Davi) and aunts (Anne, Vinciane), for showing genuine interest in my research even when I struggled to explain it in family-friendly terms.

To my cousins (Camille, Clément, Nathan, Louis, Victor, Jules, Mark, Mathilde, Adam, Jean, Matthis, Maryline, Julia, Hugo, Gabriel, Cécile, Lucie), for bringing laughter and perspective to our family gatherings.

To my parents (Pierre, Yollaine), who have supported every decision in my academic journey, never questioning my choice to pursue yet another family doctorate.

To my siblings (Marie, Emmanuel), who have masterfully balanced supporting my academic endeavors while keeping my ego in check through well-timed teasing.



