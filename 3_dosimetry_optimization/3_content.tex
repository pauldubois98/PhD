%In modern radiotherapy, the precision of dose delivery is essential to maximize the therapeutic effect on cancerous tissues while minimizing exposure to surrounding healthy organs.
%Achieving this balance relies heavily on advanced dose optimization techniques that tailor radiation to the specific anatomical and clinical needs of each patient.
\section{Discretization}
The optimization process starts with transforming the continuous nature of both the radiation field and the human body into discrete elements.
This transformation enables computation with modern computers.

\subsection[Bixels]{Fluence Map Discretization: Bixels}
Fluence maps are broken down into discrete elements called "bixels" (\textbf{be}am \textbf{el}ements).
Bixels represent small and independent beams of radiation.

The width of each bixel is constrained by the width of the multi-leaf collimator leaves.
Modern multi-leaf collimator systems typically have a leaf width of 0.5 cm.

The height of a bixel can be selected arbitrarily, as the leaf can move continuously.
Nevertheless, square bixels (akin to image pixels) are commonly used and will be employed throughout this manuscript.

Bixels whose beams do not affect the planning target volume are typically excluded from calculations to improve computational efficiency.
Activating these bixels could only degrade dose quality by increasing the dose to organs at risk without benefiting the dose distribution within the planning target volume.

\subsection[Voxels]{Human Body Discretization: Voxels}

\subsection[DI-Matrix]{Dose-Influence Matrix}

\section{Naive Method}
% least squares on PTV = prescription, OARs = 0
% => does not work at all
\section{Constraints and Importance Factors}
\subsection[DVHs]{Dose Volume Histograms}
% 3D dose vs DVH
\subsection{Constraints Formulation}
\subsection{Optimization Problem}
\paragraph{Ideal}
% minimize dose while reaching constraints
\paragraph{Practical}
% get as close as possible to meeting constraints
% building of cost function

\section{Dose Mimicking}
% principle
% => is a simpler problem
% => only works if dose to mimick is nearly acheaivable
% => can arise if machine is changing during treatment
% optim problem

\section{Optimization Algorithm Review for Dosimetry}
\subsection{Introduction}
% ArXiV paper

\subsection{Methods}
%Efficiently solving this optimization problem often involves designing the objective function to be convex, thereby providing a well-defined target for the optimization process.
%Gradient-based methods, Newtonian algorithms, or quasi-Newtonian algorithms are commonly employed for this purpose.
%We aim at benchmarking state-of-the-art open-source optimization algorithms for the specific task of radiotherapy dosimetry.

\subsection{Data}
% TG-119 \cite{AAPM-TG119}
\subsection{Objective function}
% details on the one used for this study
% => tested variations (p=1,2), changed importance factors, but v similar results
\subsection{Open-source Optimizers}
\paragraph{(Stochastic) Gradient Descent}
\paragraph{Conjugate Gradient}
\paragraph{Newton}
\paragraph{SLSQP}
\paragraph{RMSprop}
\paragraph{BFGS-based}
\subparagraph{Pure BFGS}
\subparagraph{L-BFGS}
\paragraph{Adam-based}
\subparagraph{Pure Adam}
\subparagraph{RAdam}
\subparagraph{NAdam}
\subparagraph{AdamDelta}
\subparagraph{Adamax}
\paragraph{Rprop}
\paragraph{Other optimizers variations}
%In addition, we also tested AdamW, Adagrad and ASGD.
%However, AdamW \& Adagrad behaved similarly to Adam, and ASGD behaved similarly to SGD.

\subsection{Results}
% 4 TGG-119 cases figures
\paragraph{Newton's method}
\paragraph{Best Algorithms}
\paragraph{LBFGS vs BFGS}

\subsection{Discussion}
