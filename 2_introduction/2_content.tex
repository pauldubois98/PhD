\section{Context}
Cancer remains one of the leading causes of mortality worldwide, with its incidence projected to rise in the coming decades.
As our understanding of cancer biology evolves and diagnostic techniques improve, the demand for effective and personalized treatment strategies continues to grow.
Radiotherapy has emerged as a cornerstone in cancer management, either as a standalone treatment or in combination with other modalities such as surgery, chemotherapy, and immunotherapy.
Radiotherapy leverages ionizing radiation to damage cancer cells' DNA, impeding their ability to proliferate and ultimately leading to cell death.
The efficacy of radiotherapy lies in its ability to deliver precise doses of radiation to tumor volumes while minimizing exposure to surrounding healthy tissues.
This delicate balance between tumor control and normal tissue toxicity underscores the importance of treatment planning in radiotherapy.

The advent of advanced imaging technologies, coupled with sophisticated delivery systems like Multi-Leaf Collimator Linear Accelerators (MLC-LINACs), Tomotherapy units, and CyberKnife systems, has revolutionized the field of radiation oncology.
These technological advancements have paved the way for highly conformal treatment techniques.
Intensity-Modulated Radiation Therapy (IMRT) and Volumetric Modulated Arc Therapy (VMAT) techniques offer unprecedented levels of dose sculpting, allowing for escalated doses to tumors while better sparing organs at risk.
However, the increased complexity of modern radiotherapy techniques has led to a corresponding increase in the complexity of treatment planning.
The process of creating an optimal treatment plan involves multiple steps, including Beam Orientation Optimization (BOO), Fluence Map Optimization (FMO), Leaf Sequencing (LS), and (sometimes) Direct Aperture Optimization (DAO).
Each step requires careful consideration of numerous variables and constraints, making the planning process time-consuming and labor-intensive.

Moreover, the quality of treatment plans can vary significantly based on the experience and skill of the planning team, leading to potential inconsistencies in patient care.
This variability and the growing demand for radiotherapy services have created a pressing need for more efficient and standardized planning approaches.
The automation of treatment planning processes presents a promising solution to these challenges.

Through the application of computational algorithms and artificial intelligence, automated planning systems offer the potential to significantly enhance radiotherapy treatment delivery.
By reducing planning time and increasing departmental efficiency, these systems can improve patient throughput and resource allocation.
Additionally, automated planning can lead to improved plan quality and consistency, reducing variability and ensuring optimal treatment outcomes.
Furthermore, the ability to enable rapid re-planning facilitates adaptive radiotherapy, allowing for adjustments to treatment plans in response to changes in tumor volume or patient anatomy.
Finally, automated systems can explore a more expansive solution space, potentially leading to the discovery of novel and innovative planning strategies that may improve treatment outcomes.

However, the development and implementation of automated planning systems pose challenges.
These challenges include the creation of robust optimization algorithms, integrating with existing Treatment Planning Systems, and validating against current clinical standards.

In this context, this thesis aims to explore and advance the radiotherapy treatment planning automation field, focusing on developing novel algorithms and methodologies to enhance the efficiency, quality, and consistency of treatment plans.
By building upon the foundational knowledge of radiotherapy physics, biology, and clinical workflow, we seek to contribute to the ongoing evolution of radiation oncology and, ultimately, to improve outcomes for cancer patients.

\section{Problematic}
Traditional manual radiotherapy planning procedures are inherently subjective and time-consuming, often leading to variability in treatment plan quality due to the reliance on individual planner expertise and experience.
While efforts have been made to enhance consistency, significant variability among planners and institutions persists.

\section{State of the Art}
\paragraph{Knowledge-based radiotherapy planning}

\paragraph{Conventional techniques}

\paragraph{Pareto surface exploration}

\section{Unsolved problems}

\section{Contribution}
