% Content of the chapter 1 of my PhD thesis
\section{Medical context}

This PhD thesis is about radiation therapy (RT) for cancer treatment.

\subsection{About cancer}

Cancer is a complex disease that can affect many part of the body.
It is characterized by the uncontrolled growth of cells that can invade and destroy surrounding tissues.
Cancer is a leading cause of death worldwide.
The World Health Organization (WHO) estimated 20 million new cancer cases in 2022, and 9.6 million deaths linked to cancer in 2022\cite{who_cancer2022}.
Cancer touches about 20\% of the population, and is responsible for 1 in 10 deaths.

\paragraph{Cancer markers}
There are several cancer markers.
Cancer cells proliferate uncontrollably.
They also reprogram cellular metabolism to support their growth\cite{Chammas2013}.
They can also stop cell growth arrest mechanisms.
They usually manage to evade apoptosis (the programmed cell death).
Cancer cells can escape the immune system, and change their cellular response phenotypic via plasticity.
At some point, cancer cell can get the ability to undergo a sufficient number of successive cell cycles of growth and division to generate macroscopic tumors.
To support their growth, they create new blood vessels to get nutriments.
Finally, they can escape and form metastasis, and will eventually provoke senescence.

\paragraph{Conditions leading to cancer}
Cancer is a complex disease.
First, cancer is caused by mutations in the DNA.
These mutations can be inherited or acquired.
Second, cancer embrased by an epigenetic reprogramming.
That is, changes in gene expression that are not caused by changes in the DNA sequence.
Third, cancer is often associated with an inflammatory context.
Inflammation can promote cancer growth and spread.
Finally, cancer is often associated with a disruption of the microbiota.
The microbiota is the community of microorganisms that live in and on the human body.
Disruption of the microbiota can promote cancer growth and spread.

\paragraph{Phases of cancer}
Cancer devolops in several phases.
\subparagraph{Initiation}
The first phase is initiation.
During this phase, a normal cell is transformed into a cancer cell.
This transformation is caused by mutations in the DNA.
\subparagraph{Promotion and tumorigenesis}
The second phase is promotion/tumorigenesis.
During this phase, the cancer cell begins to grow and divide uncontrollably to form a cluster of cells called tumor.
This growth is promoted by changes in gene expression and other factors\cite{Witsch2010}.
It may also creates new blood vessels to get nutriments and oxygen.
\subparagraph{Evolution}
The final phase is evolution.
The tumor will first grow locally, then regionally, invading and damaging surrounding tissues.
Finally, the cancer cell will spread to other parts of the body, forming metastasis.
Metastasis is the main cause of death in cancer patients\cite{Steeg2006}.

\subparagraph{Cancer stages}
Cancer is classified into stages\cite{cancerresearchuk2023}.
\begin{itemize}
	\item Stage 0: 'in situ neoplasm'; it means that there is a group of abnormal cells in an area of the body. The cells may develop into cancer at some time in the future.
	\item Stage 1: the cancer is small and contained within the organ it started in.
	\item Stage 2: the tumour is larger than in stage 1 but the cancer hasn't started to spread into the surrounding tissues.
	\item Stage 3: the cancer is larger, it have started to spread into surrounding tissues and there are cancer cells in the lymph nodes nearby.
	\item Stage 4: the cancer has spread from where it started to another body organ. This is also called secondary or metastatic cancer.
\end{itemize}
Doctors use the TNM system to describe the stage of the cancer\cite{Brierley2016}.
\begin{itemize}
	\item [T] stands for the size of the Tumour;
	It can be 1, 2, 3 or 4, with 1 being small and 4 large.
	\item [N] stands for the number of lymph Nodes affected;
	It can be between 0 and 3.
	0 means that there are no lymph nodes containing cancer cells; 3 means that there are lots of lymph nodes containing cancer cells.
	\item [M] stands for existance of metastasis in another part of the body.
	It can be 0 (no spread) or 1 (the cancer has spread).
\end{itemize}

\paragraph{Most common cancers}
According to the WHO, the most common cancers are lung, breast, colorectal, prostate, skin, and stomach cancer.
This thesis mainly focus on prostate cancer, which is among the most common cancers.

\paragraph{Risk factors}
Tobacco use, alcohol consumption, unhealthy diet, physical inactivity and air pollution are risk factors for other cancer types.
However, the main risk factor for prostate cancer is age.

\subsection{Treatment types}

There are three main types of cancer treatment: surgery, radiation therapy, and chemotherapy.


\paragraph{Surgery}
Surgery is the most common treatment for cancer.
It involves removing the tumor and surrounding tissue.
Surgery is often used to treat early-stage cancer that has not spread to other parts of the body.
For surgery to be possible, the tumor must be located in a place that can be easily accessed by the surgeon.
Surgery can be followed by other treatments, such as radiation therapy or chemotherapy, to kill any remaining cancer cells.
% give TNM / stage of cancer where surgery is possible
% give success rate of surgery

\subparagraph{Advantages}
Surgery is curative, meaning that cancer is completely removed and the patient can "forget" about it.
It is also a local treatment, hence having limited side effects on the body.
Finally, only one session is needed.

\subparagraph{Disadvantages}
Surgery is invasive, and can be painful.
It can only be used for cancer that are localized, (with no metastasis), and accessible for the surgeon.
Finally, it can be expensive.


\paragraph{surgery}
\paragraph{RT}
% DNA quality plot with DNA reparation only for healthy cells
\paragraph{chemotherapy}
\paragraph{combination}

% %%%%%%%%%%%%%%%%%%%%%%%%%%%%%%%%%%%%%%%%%%%%%%%%%%%% %
\section{Patient Path}

% add a graphic
\subsection{Detection / diagnostic}
\subsection{RT Prescription}
\subsection{CT scan}
\subsection{Contouring}
\subsection{Treatment Planning}
\subsection{Irradiation Sessions}
\subsection{Follow-up}

\section{Machines}
% add graphic of usage per machine type, and per constructor
\subsection{Molds / 3D-RT}
\subsection{MLC-LINAC}
% modifiers: jaws & leaves
\subsection{Tomotherapy}
\subsection{CyberKnife}
\subsection{Brachytherapy}

\section{Irradiations techniques}
\subsection{IMRT}
\paragraph{Step and Shoot}
\paragraph{Sliding Window}
% => what we will focus on
\subsection{VMAT}

\section{Treatment Planning Systems}
\subsection{Manufacturer}
\paragraph{Eclipse (Varian)}
\paragraph{ONE | Planning (Elekta)}
\paragraph{Precision (Accuray)}
\subsection{Non-manufacturer}
\paragraph{RayStation (RaySearch)}
\paragraph{matRad (German Cancer Research Center - DKFZ)}
\paragraph{AutoPlan (TheraPanacea - coming soon)}

\section{Dosimetry steps}
\paragraph{Challenges}
% competeing goals
\subsection{BOO}
\subsection{FMO}
\subsection{LF}

\section{Simulation}
% https://oncologymedicalphysics.com/dose-calculation-algorithms/
