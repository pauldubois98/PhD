% title page
\def \englishTitle {Methods for automatization of radiotherapy dosimetry.}
\def \frenchTitle {Méthodes pour l'automatisation de la dosimetrie en radiothérapie.}

\def \specialty {Spécialité de doctorat: ...}
\def \doctoralSchool {École doctorale n$^{\circ}$ 573 Interfaces : matériaux, systèmes, usages, ED INTERFACE}
\def \graduateSchool {Graduate School: Sciences de l’Ingénierie et des Systèmes, SIS}

\def \researchUnits {
	\textbf{Radiothérapie} (Institut Régionale du Cancer de Montpellier), \textbf{Advanced Research} (TheraPanacea), et \textbf{MICS, Mathématiques et Informatique pour la Complexité et les Systèmes} (Université Paris-Saclay, CentraleSupélec)
}
\def \directors {
	\textbf{Nikos Paragios}, Professeur, et la co-direction de \textbf{Paul-Henry Cournède}, Professeur
}

\def \date {JJ mois AAAA}
\def \author{Paul Raymond François DUBOIS}

\def \jury {
	\textbf{Prénom NOM} & Président ou Présidente\\
	Titre, Affiliation & \\
	\textbf{Prénom NOM} & Rapporteur \& Examinateur / trice \\
	Titre, Affiliation & \\
	\textbf{Prénom NOM} & Rapporteur \& Examinateur / trice \\
	Titre, Affiliation & \\
	\textbf{Prénom NOM} & Examinateur ou Examinatrice \\
	Titre, Affiliation & \\
	\textbf{Prénom NOM} &  Examinateur ou Examinatrice \\
	Titre, Affiliation & \\
}

% abstract page
\def \englishKeyWords {Mathematics, Artificial Intelligence, Radiotherapy}
\def \frenchKeyWords {Mathématiques, Intelligence Artificielle, Radiothérapie}

\def \englishAbstract {
	\lipsum[1-2]
}

\def \frenchAbstract {
	La dosimétrie en radiothérapie est essentielle pour garantir la précision et la sécurité des traitements contre le cancer.
	La complexité et la variabilité de la planification des traitements nécessitent des méthodologies avancées pour l'automatisation et l'optimisation.
	Cette thèse présente des approches novatrices visant à automatiser le processus de dosimétrie en radiothérapie.
	
	Cette thèse commence par le développement d'un moteur de dosimétrie et une évaluation approfondie des algorithmes d'optimisation open-source existants pour la planification des traitements.
	Ensuite, ce manuscrit analyse les relations entre différentes doses.
	Cette analyse conduit à la proposition d'un cadre novateur pour l'optimisation multi-objectif et la sélection robuste de plans à l'aide de la théorie des graphes.
	
	Afin de réduire davantage le temps nécessaire pour la planification en radiothérapie, la thèse explore l'application de l'apprentissage par renforcement pour l'optimisation des doses.
	Le système proposé réalise la dosimétrie pour de nouveaux patients en exploitant les données de dose des patients traités dans le passé.
	Cette méthode entièrement automatisée peut s'adapter aux pratiques de différentes cliniques, réduisant ainsi le besoin d'ajustements manuels et facilitant son adoption en pratique.
	
	De plus, la thèse examine l'utilisation de l'apprentissage profond pour la prédiction des doses, en proposant une série de modèles guidés par des Histogrammes Dose-Volume (DVH) cibles.
	Ce guidage orientation permet l'incorporation de directives lors de la génération de doses par les modèles.
	En outre, cette technique permet d'entraîner un seul modèle capable de s'adapter, plutôt qu'un modèle pour chaque clinique.
	
	Les contributions de cette thèse présentent des avancées dans la dosimétrie en radiothérapie, ouvrant la voie au développement d'un système de planification de traitement entièrement automatisé, s'adaptant aux contraintes cliniques.
	, conçu pour fonctionner avec une e.
	Ces innovations pourraient améliorer les flux de travail cliniques, en réduisant l'intervention humaine à un minimum, rendant la radiothérapie plus efficiente.
}